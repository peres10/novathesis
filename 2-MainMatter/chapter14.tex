%!TEX root = ../template.tex
%%%%%%%%%%%%%%%%%%%%%%%%%%%%%%%%%%%%%%%%%%%%%%%%%%%%%%%%%%%%%%%%%%%%
%% chapter14.tex
%% NOVA thesis document file
%%
%% Chapter with lots of dummy text
%%%%%%%%%%%%%%%%%%%%%%%%%%%%%%%%%%%%%%%%%%%%%%%%%%%%%%%%%%%%%%%%%%%%

\typeout{NT FILE chapter14.tex}%

\chapter{Troubleshooting and Frequently Asked Questions}
\label{chap:troubleshooting}

\section{Overview}

This chapter provides solutions to the most common issues encountered when compiling, configuring, or customizing the \gls{novathesis} template.  
It also answers frequently asked questions regarding supported environments, fonts, language options, and institutional presets.  
Each subsection addresses a distinct category of problems, followed by a concise resolution procedure.

\section{Compilation Problems}

\subsection{The document fails to compile on the first run}

\paragraph{Symptom.}  
\texttt{latexmk} or \texttt{make} aborts with “undefined references” or “missing file” errors on a clean installation.

\paragraph{Cause.}  
The auxiliary files required by \LaTeX{} (e.g., \texttt{.aux}, \texttt{.toc}, \texttt{.bbl}) do not exist yet.

\paragraph{Resolution.}  
Run the full build command twice:
\begin{verbatim}
make xe
make xe
\end{verbatim}
or simply use \texttt{latexmk}, which automatically performs multiple passes:
\begin{verbatim}
latexmk -xelatex template.tex
\end{verbatim}
All cross-references, tables, and bibliography entries should then resolve correctly.

\subsection{Bibliography not appearing or citations show as “(?)”}

\paragraph{Cause.}  
The bibliography backend (\texttt{biber} or \texttt{bibtex}) was not executed.

\paragraph{Resolution.}
\begin{enumerate}
  \item Ensure that the backend specified in \texttt{0-Config/2\_biblatex.tex} matches the installed tool:
  \begin{verbatim}
  \ntbibsetup{backend=biber}
  \end{verbatim}
  \item Recompile cleanly:
  \begin{verbatim}
  make clean
  make xe
  \end{verbatim}
  \item If errors persist, run manually:
  \begin{verbatim}
  biber template
  latexmk -xelatex template.tex
  \end{verbatim}
\end{enumerate}

\paragraph{Additional note.}  
In Overleaf, the bibliography backend must be set to \texttt{biber} under the project menu.  
Free accounts may not support this due to resource limits (see Section~\ref{sec:overleaf-limitations}).

\subsection{Font not found or substituted}

\paragraph{Symptom.}  
Warnings such as “Font not found: Arial” or “Font substitution: CM Roman used instead.”

\paragraph{Cause.}  
The selected font theme requires a system or OpenType font that is not installed.

\paragraph{Resolution.}
\begin{itemize}
  \item Install the required font family on the system;
  \item Recompile using XeLaTeX or LuaLaTeX;
  \item If using Overleaf, switch to a fully portable font theme (e.g., \texttt{newpx}, \texttt{libertine}, or \texttt{erewhon}).
\end{itemize}

\paragraph{Tip.}  
To verify which font theme is active, check the line
\begin{verbatim}
\ntsetup{style/font=<name>}
\end{verbatim}
in \texttt{0-Config/1\_novathesis.tex}.

\subsection{Missing packages or “Undefined control sequence” errors}

\paragraph{Cause.}  
Your \LaTeX{} distribution lacks one or more required packages.

\paragraph{Resolution.}
\begin{itemize}
  \item For macOS or Linux, install the full distribution:
  \begin{verbatim}
  sudo apt install texlive-full
  \end{verbatim}
  or reinstall MacTeX from \url{https://tug.org/mactex/}.
  \item For Windows, open the MiKTeX Console and enable “Install missing packages on the fly.”
\end{itemize}

\paragraph{Diagnosis.}  
Inspect the first missing package name reported in the log, and install it manually if necessary.  
For example:
\begin{verbatim}
tlmgr install titlesec
\end{verbatim}

\subsection{Glossaries or acronyms not generated}

\paragraph{Cause.}  
The glossaries build step was not executed.

\paragraph{Resolution.}
\begin{enumerate}
  \item Ensure that the relevant glossaries packages are loaded in \texttt{0-Config/5\_packages.tex};
  \item Add the corresponding list in \texttt{0-Config/6\_list\_of.tex}:
  \begin{verbatim}
  \ntaddlistof{listsofglossaries}
  \end{verbatim}
  \item Perform a full rebuild:
  \begin{verbatim}
  make clean-all
  make xe
  \end{verbatim}
\end{enumerate}

\subsection{Figures not displaying}

\paragraph{Cause.}  
File path or format mismatch.

\paragraph{Resolution.}
\begin{itemize}
  \item Place all figures under \texttt{5-Figures/};
  \item Use relative paths (e.g., \verb|\includegraphics{5-Figures/figure1}|);
  \item Prefer PDF or PNG formats; avoid JPEG for line art;
  \item Ensure that capitalization matches exactly—paths are case-sensitive on Linux and macOS.
\end{itemize}

\subsection{Compilation too slow or running out of memory}

\paragraph{Possible causes.}
\begin{itemize}
  \item Large images with excessive resolution;
  \item Multiple redefinitions of macros in \texttt{5\_packages.tex};
  \item Overuse of TikZ or complex vector graphics.
\end{itemize}

\paragraph{Mitigation.}
\begin{itemize}
  \item Downsample images to 300\,dpi for print-quality output;
  \item Comment unused packages in \texttt{5\_packages.tex};
  \item Use externalized TikZ figures (\texttt{tikzexternalize}) to compile them once.
\end{itemize}

\section{Overleaf-Specific Issues}
\label{sec:overleaf-limitations}

\subsection{Project does not compile in Overleaf}

\paragraph{Cause.}  
Free-tier Overleaf accounts provide limited memory and compilation time. \gls{novathesis} is a large, multi-pass template.

\paragraph{Resolution.}
\begin{itemize}
  \item Use a \textbf{Professional} or \textbf{Group Plan} account, which increases build resources;
  \item Set the compiler to XeLaTeX or LuaLaTeX;
  \item Enable the \texttt{biber} backend under “Project Settings”;
  \item Clean auxiliary files manually if builds fail repeatedly.
\end{itemize}

\paragraph{Alternative.}  
Compile locally using \texttt{make xe}, then upload only the final PDF to Overleaf for collaboration.

\subsection{Images or logos not appearing in Overleaf}

\paragraph{Cause.}  
Overleaf ignores symbolic links and certain non-ASCII file names.

\paragraph{Resolution.}
\begin{itemize}
  \item Ensure all logos and figures are actual files (no symlinks);
  \item Avoid characters with accents or spaces in filenames;
  \item Use plain ASCII filenames (\texttt{a-z0-9\_}) and relative paths.
\end{itemize}

\subsection{Bibliography errors in Overleaf}

\paragraph{Cause.}  
Biber execution timeout or memory limit reached.

\paragraph{Resolution.}
\begin{itemize}
  \item Reduce bibliography size (split large \texttt{.bib} files);
  \item Recompile locally to generate \texttt{.bbl}, then upload that file to Overleaf;
  \item Contact Overleaf support if the issue persists with a Professional account.
\end{itemize}

\section{Institutional Preset and Cover Issues}

\subsection{Institution logo missing on cover}

\paragraph{Cause.}  
The preset references a logo file that does not exist in \texttt{\gls{novathesis}Files/Logos/}.

\paragraph{Resolution.}
\begin{itemize}
  \item Confirm the expected filename (e.g., \texttt{nova\_fct.pdf});
  \item Ensure the extension is lowercase (\texttt{.pdf});
  \item Replace the logo file or edit the preset to use the correct name:
  \begin{verbatim}
  \ntsetup{logo/front=mylogo}
  \end{verbatim}
\end{itemize}

\subsection{Wrong institution name or degree shown}

\paragraph{Cause.}  
Multiple degree definitions are uncommented in the preset file.

\paragraph{Resolution.}
Open the preset file (e.g., \texttt{0-Config/9\_nova\_fct.tex}) and ensure that only one pair of \verb|\ntdegreename(pt)| and \verb|\ntdegreename(en)| is uncommented.

\subsection{Committee page shows missing names}

\paragraph{Cause.}  
Committee member entries were not defined in \texttt{0-Config/3\_cover.tex}.

\paragraph{Resolution.}
Ensure each committee member is declared using the syntax:
\begin{verbatim}
\ntaddperson{committee}(r,m){Dr. John Doe, Associate Professor, NOVA}
\end{verbatim}
where the first parameter indicates the role and the second indicates gender (m/f).

\section{Output Formatting Issues}

\subsection{Unexpected spacing or indentation}

\paragraph{Cause.}  
A \texttt{\textbackslash parskip} or \texttt{\textbackslash baselinestretch} command was redefined in a user package.

\paragraph{Resolution.}
Check \texttt{0-Config/5\_packages.tex} and comment out any spacing or layout packages such as \texttt{setspace}, \texttt{parskip}, or \texttt{titlesec}.  
\gls{novathesis} manages these internally through \texttt{memoir}.

\subsection{Incorrect page numbering}

\paragraph{Cause.}  
Improper file order or mixed numbering styles.

\paragraph{Resolution.}
Ensure that front matter files (abstracts, acknowledgments) appear before main matter chapters in \texttt{0-Config/4\_files.tex}.  
The class automatically resets numbering at the transition between front and main matter.

\section{Advanced Debugging Techniques}

\subsection{Using the \texttt{--interaction=nonstopmode} flag}

Run the compiler in nonstop mode to suppress interactive prompts:
\begin{verbatim}
latexmk -xelatex -interaction=nonstopmode template.tex
\end{verbatim}
This is the mode used internally by \texttt{make xe}.  
All errors will be logged to \texttt{template.log}.

\subsection{Examining the Log File}

Search the log for:
\begin{itemize}
  \item \texttt{! LaTeX Error:} — shows the exact command causing failure;
  \item \texttt{Missing \textbackslash end\{...}} — indicates mismatched environments;
  \item \texttt{File ... not found} — missing figures or bibliography files.
\end{itemize}

\paragraph{Tip.}  
Use a text editor with syntax highlighting and search (e.g., VS Code, TeXstudio, or TeXworks).

\subsection{Verbose bibliography debugging}

Run:
\begin{verbatim}
biber --debug template
\end{verbatim}
to obtain detailed information about the bibliography parsing process.  
Inspect the \texttt{template.blg} file for missing fields or syntax errors in the \texttt{.bib} database.

\subsection{Minimal Working Example (MWE)}

When requesting support, isolate the issue by creating a minimal working example:
\begin{enumerate}
  \item Copy \texttt{template.tex} and include only one configuration file and one chapter;
  \item Remove all images and external packages;
  \item Test compilation;
  \item Gradually reintroduce content until the error reappears.
\end{enumerate}

\section{Frequently Asked Questions}

\subsection*{Q1: Can I use \gls{novathesis} for non-NOVA universities?}

Yes.  
While the template was originally developed for NOVA University Lisbon, it is fully modular.  
You can create new presets in \texttt{0-Config/9\_<school>.tex} (see Chapter~\ref{sec:new-school}).

\subsection*{Q2: Does \gls{novathesis} support double-spacing?}

Yes.  
Add to \texttt{0-Config/0\_memoir.tex}:
\begin{verbatim}
\DoubleSpacing
\end{verbatim}
However, confirm that your institution permits double-spacing in the final submission.

\subsection*{Q3: Can I print on A4 and Letter paper interchangeably?}

Yes.  
Set the paper size through your TeX distribution or in the document class options:
\begin{verbatim}
\documentclass[a4paper]{novathesis}
\end{verbatim}
For US submissions, use \texttt{letterpaper}.

\subsection*{Q4: Is XeLaTeX better than LuaLaTeX?}

Both engines produce high-quality Unicode output.  
XeLaTeX offers simpler font handling; LuaLaTeX provides finer microtypography control.  
Most users will not notice practical differences.

\subsection*{Q5: How do I include code listings?}

Load either \texttt{listings} or \texttt{minted} in \texttt{5\_packages.tex}.  
For \texttt{minted}, compile with:
\begin{verbatim}
make xe LATEXMKOPTS="--shell-escape"
\end{verbatim}

\subsection*{Q6: Can I use multiple bibliography files?}

Yes.  
Add them to \texttt{0-Config/4\_files.tex}:
\begin{verbatim}
\ntaddfile{bib}{main.bib}
\ntaddfile{bib}{secondary.bib}
\end{verbatim}
\gls{novathesis} merges them automatically in the final bibliography.

\subsection*{Q7: How can I disable colored hyperlinks?}

Set in \texttt{1\_novathesis.tex}:
\begin{verbatim}
\ntsetup{media=paper}
\end{verbatim}
This changes hyperlink colors to black and adjusts margins for print output.

\subsection*{Q8: How can I check my \gls{novathesis} version?}

Open \texttt{novathesis.cls} and look for the line:
\begin{verbatim}
\ProvidesClass{novathesis}[YYYY/MM/DD vX.Y.Z]
\end{verbatim}
or check the header of \texttt{README.md}.  
Document the version in your thesis metadata for reproducibility.

\subsection*{Q9: How can I change numbering style (e.g., from Arabic to Roman)?}

Add the following in \texttt{0-Config/0\_memoir.tex}:
\begin{verbatim}
\renewcommand{\thechapter}{\Roman{chapter}}
\end{verbatim}
Avoid redefining numbering directly in \texttt{novathesis.cls}.

\subsection*{Q10: How do I report bugs or request new features?}

Submit an issue to the official repository or contact the maintainer through your institution’s documentation portal.  
Include:
\begin{itemize}
  \item The \gls{novathesis} version number;
  \item The TeX distribution and engine used;
  \item A minimal working example reproducing the problem.
\end{itemize}

\section{Summary}

This chapter has presented comprehensive solutions to common compilation, configuration, and formatting problems encountered when using \gls{novathesis}.  
By following the structured troubleshooting procedures and verifying system setup, users can resolve nearly all issues without modifying the class itself.  
For persistent or institutional-specific problems, users should isolate the issue with a minimal example and consult the maintainers with full version and environment details.

