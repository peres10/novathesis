%!TEX root = ../template.tex

\chapter{Project Maintenance and Future Development}
\label{chap:maintenance}

\section{Overview}

This chapter documents the long-term maintenance strategy and development roadmap for the \gls{novathesis} Template.  
It is intended for maintainers, institutional partners, and contributors responsible for ensuring the continuity, stability, and evolution of the project.

\gls{novathesis} has matured into a modular and extensible platform that serves as the standard academic writing framework across NOVA University Lisbon and other partner institutions.  
To preserve its long-term relevance, the project adheres to structured maintenance practices, semantic versioning, and formal governance procedures.

\section{Maintenance Objectives}

The main objectives of project maintenance are:

\begin{enumerate}
  \item \textbf{Reliability} — ensuring that every release compiles successfully across supported systems;
  \item \textbf{Compatibility} — preserving backward compatibility of configuration files and institutional presets;
  \item \textbf{Compliance} — maintaining alignment with institutional formatting and publication standards;
  \item \textbf{Sustainability} — providing a clear development roadmap, documentation updates, and succession planning;
  \item \textbf{Reproducibility} — ensuring that older theses can be recompiled identically using the same template version.
\end{enumerate}

\section{Maintenance Model}

\gls{novathesis} follows a hybrid maintenance model combining centralized coordination with community input.  

\paragraph{Core Maintainers.}
A small group of maintainers is responsible for:
\begin{itemize}
  \item Reviewing and merging contributions;
  \item Coordinating institutional preset updates;
  \item Publishing release tags and build artifacts;
  \item Maintaining documentation and build automation.
\end{itemize}

\paragraph{Community Contributors.}
External contributors may:
\begin{itemize}
  \item Submit bug reports and feature proposals;
  \item Contribute institutional styles and translations;
  \item Improve documentation and example projects;
  \item Participate in roadmap discussions.
\end{itemize}

All development activity occurs through transparent processes documented in the public repository.

\section{Release Lifecycle Policy}

Each major \gls{novathesis} release is maintained under the following lifecycle:

\begin{itemize}
  \item \textbf{Active Maintenance:} 12 months — includes feature updates and bug fixes;
  \item \textbf{Extended Support:} 12 additional months — security and compatibility fixes only;
  \item \textbf{Archived Status:} after 24 months — release remains accessible but unsupported.
\end{itemize}

Maintenance versions are tracked in the \texttt{CHANGELOG.md} file.  
Older releases remain permanently available in the GitHub Releases archive to ensure reproducibility of previously submitted theses.

\section{Automated Testing and Continuous Integration}

Each commit and pull request undergoes automated testing to verify compilation integrity.

\paragraph{Testing Pipeline.}
\begin{itemize}
  \item Compilation using \texttt{latexmk} under \texttt{pdfLaTeX}, \texttt{XeLaTeX}, and \texttt{LuaLaTeX};
  \item Validation of glossary, bibliography, and cross-references;
  \item Verification of institutional presets and multi-language examples.
\end{itemize}

Failures must be resolved before merging into the \texttt{main} branch.  
Testing infrastructure is implemented using GitHub Actions or an equivalent CI environment.

\section{Documentation Maintenance}

The \gls{novathesis} manual (this document) is versioned alongside the template and updated for every minor release.  
Documentation is maintained in both \texttt{.tex} and \texttt{.pdf} formats, and all examples are verified through automated builds.

\paragraph{Documentation Sources.}
\begin{itemize}
  \item \texttt{docs/manual/} — \LaTeX{} sources of the User Manual;
  \item \texttt{README.md} — short introduction for repository users;
  \item \texttt{examples/} — compilable sample projects illustrating usage.
\end{itemize}

\paragraph{Publication.}
Updated documentation is distributed with every tagged release and uploaded to the project’s GitHub Pages site for online access.

\section{Dependency Management}

To minimize incompatibility with upstream \LaTeX{} distributions:
\begin{itemize}
  \item Dependencies are limited to core packages included in TeX~Live, MacTeX, or MiKTeX;
  \item Version constraints are documented in \texttt{\gls{novathesis}Files/requirements.txt};
  \item The project is periodically tested against the latest TeX~Live distribution (LTS and rolling versions).
\end{itemize}

If upstream packages introduce breaking changes, maintainers issue compatibility patches under the next minor version.

\section{Institutional Collaboration}

\gls{novathesis} encourages formal collaboration with universities and research institutions.  
Each partner institution may contribute:
\begin{itemize}
  \item Custom visual identity presets (cover pages, logos, color schemes);
  \item Localized string files for additional languages;
  \item Standardized formatting policies for specific degree programs.
\end{itemize}

To ensure consistency, new institutional presets must undergo review by the core maintainers before inclusion in the main repository.

\section{Internationalization and Accessibility Roadmap}

\paragraph{Internationalization.}
Future versions aim to expand built-in localization support beyond the existing languages, with improved handling of right-to-left scripts and additional regional typographic standards.

\paragraph{Accessibility.}
Planned improvements include:
\begin{itemize}
  \item Automatic tagging of structural elements for screen readers (PDF/UA compliance);
  \item Better contrast verification for color themes;
  \item Guidelines for accessible figure captions and table descriptions.
\end{itemize}

\section{Technical Roadmap}

The following roadmap outlines the current development priorities:

\begin{description}
  \item[Version 7.7.0] — Improved build performance, modular bibliography styles, and expanded example projects.
  \item[Version 7.8.0] — Enhanced institutional preset management and internationalized metadata fields.
  \item[Version 8.0.0] — Major release introducing complete class modularization, Lua-based configuration, and native PDF/A generation.
\end{description}

Each roadmap item is subject to revision following community review and institutional requirements.

\section{Sustainability and Archival Strategy}

Long-term sustainability depends on transparent documentation, reproducible builds, and open governance.  
The following practices are enforced:

\begin{itemize}
  \item All releases archived with digital signatures and SHA-256 checksums;
  \item Documentation preserved in PDF/A format for institutional libraries;
  \item Source code and manual mirrored in the university’s GitLab and GitHub instances;
  \item A stable identifier (DOI) assigned to each major version for citation.
\end{itemize}

\section{Succession Planning}

To guarantee project continuity:
\begin{itemize}
  \item At least two maintainers must have administrative access to the repository;
  \item Maintenance credentials are documented in the institutional repository;
  \item Handover documentation describes the build process, test routines, and release workflow;
  \item All core macros and configuration files include inline version metadata.
\end{itemize}

\section{Community Engagement}

Community engagement is central to \gls{novathesis} sustainability.  
The maintainers encourage:
\begin{itemize}
  \item Academic workshops and thesis preparation seminars using the template;
  \item Student contributions through open coursework or theses;
  \item Continuous feedback via GitHub Discussions;
  \item Cross-institutional collaboration for formatting harmonization.
\end{itemize}

Community input directly informs roadmap priorities and design decisions.

\section{Acknowledgment of Institutional Support}

NOVA University Lisbon and its affiliated schools provide infrastructure and academic oversight for \gls{novathesis}.  
Special recognition is extended to the faculty members and students who contributed to the early design, testing, and translation of the template.  
Institutional support ensures that the project remains aligned with current academic publishing standards.

\section{End of Document Notice}

This User Manual, version~7.6.0, concludes the official documentation for the corresponding \gls{novathesis} release.  
Readers are encouraged to verify the version date printed on the cover and consult the repository for subsequent updates.

\paragraph{Citation.}
\begin{quote}
NOVA University Lisbon. \emph{\gls{novathesis} Template and User Manual}.  
Version~7.6.0, April~2025. Available at  
\url{https://github.com/novathesis/novathesis}.
\end{quote}

\paragraph{Contact.}
For official correspondence and maintenance coordination:
\begin{quote}
\textbf{\gls{novathesis} Coordination Team}\\
NOVA University Lisbon\\
Lisbon, Portugal\\
\url{https://novathesis.github.io}
\end{quote}

\section{Summary}

This chapter defines the long-term maintenance and development framework of the \gls{novathesis} project.  
By maintaining rigorous testing, transparent governance, and institutional collaboration, \gls{novathesis} ensures durability and adaptability in the evolving academic and technological landscape.  
Through sustained community engagement and adherence to open standards, the project will continue to support the preparation of high-quality academic theses for years to come.

