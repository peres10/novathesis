%!TEX root = ../template.tex
%%%%%%%%%%%%%%%%%%%%%%%%%%%%%%%%%%%%%%%%%%%%%%%%%%%%%%%%%%%%%%%%%%%%
%% chapter11.tex
%% NOVA thesis document file
%%
%% Chapter with lots of dummy text
%%%%%%%%%%%%%%%%%%%%%%%%%%%%%%%%%%%%%%%%%%%%%%%%%%%%%%%%%%%%%%%%%%%%

\typeout{NT FILE chapter11.tex}%

\chapter{Equations, Algorithms, and Mathematical Formatting}
\label{chap:math}

\section{Overview}

Mathematical notation, algorithms, and formal expressions are critical components of technical and scientific writing.  
\gls{novathesis} inherits the powerful mathematical capabilities of \LaTeX{} and extends them through the \texttt{amsmath}, \texttt{amsthm}, and \texttt{algorithm} packages.  
This chapter explains how to typeset equations, define theorem environments, and present algorithms in a clear and consistent manner aligned with institutional standards.

\section{Mathematics Packages in \gls{novathesis}}

The following packages are preloaded by default:

\begin{itemize}
  \item \texttt{amsmath}, \texttt{amssymb}, and \texttt{amsfonts} — core AMS packages for advanced mathematical typesetting;
  \item \texttt{mathtools} — an extension of \texttt{amsmath} providing enhanced alignment and tagging;
  \item \texttt{siunitx} — formatting of physical units and numerical quantities;
  \item \texttt{bm} — bold math symbols;
  \item \texttt{algorithm}, \texttt{algorithmicx}, and \texttt{algpseudocode} — for presenting pseudocode algorithms.
\end{itemize}

Users may load additional math packages in \texttt{0-Config/5\_packages.tex} as needed, provided they are compatible with \texttt{memoir}.

\section{Inline and Display Equations}

Equations may appear inline or as separate numbered displays.

\paragraph{Inline Example.}
\begin{verbatim}
The energy is defined by $E = mc^2$.
\end{verbatim}

\paragraph{Display Example.}
\begin{verbatim}
\begin{equation}
E = mc^2
\label{eq:energy}
\end{equation}
\end{verbatim}

\paragraph{Referencing Equations.}
\begin{verbatim}
As shown in Equation~\ref{eq:energy}, energy and mass are equivalent.
\end{verbatim}

Cross-references automatically include the localized label (“Equation”, “Equação”, etc.).

\section{Aligning Equations}

For systems of equations or multi-line derivations, use the \texttt{align} or \texttt{aligned} environments from \texttt{amsmath}.

\paragraph{Example.}
\begin{verbatim}
\begin{align}
a^2 + b^2 &= c^2 \label{eq:pythagoras}\\
E &= mc^2  \label{eq:einstein}
\end{align}
\end{verbatim}

Each line may be independently numbered and referenced.

\paragraph{Suppressing Numbers.}
Add the \verb|\nonumber| command or use the \texttt{align*} environment to omit numbering:
\begin{verbatim}
\begin{align*}
x' &= Ax + Bu \\
y  &= Cx + Du
\end{align*}
\end{verbatim}

\section{Equation Numbering and Formatting}

By default, \gls{novathesis} numbers equations sequentially within each chapter:
\begin{verbatim}
(2.1), (2.2), ...
\end{verbatim}

To number equations continuously throughout the document, modify \texttt{0-Config/0\_memoir.tex}:
\begin{verbatim}
\counterwithout{equation}{chapter}
\end{verbatim}

\paragraph{Custom Equation Format.}
For institutions requiring specific numbering formats (e.g., section-based), redefine:
\begin{verbatim}
\numberwithin{equation}{section}
\end{verbatim}

\section{Multiline Equations}

For long expressions, use \texttt{multline} or \texttt{split}:

\begin{verbatim}
\begin{multline}
f(x) = x^3 + 2x^2 + 3x + 5 \\
+ 7x^{-1} + 9x^{-2}
\end{multline}
\end{verbatim}

This environment breaks equations automatically across multiple lines while maintaining alignment and numbering.

\section{Matrices and Vectors}

Matrices are supported through \texttt{amsmath} environments.

\paragraph{Example.}
\begin{verbatim}
\begin{equation}
A = 
\begin{bmatrix}
  a_{11} & a_{12} & a_{13} \\
  a_{21} & a_{22} & a_{23}
\end{bmatrix}
\end{equation}
\end{verbatim}

\paragraph{Alternate Bracket Styles.}
\texttt{pmatrix} for parentheses, \texttt{vmatrix} for determinants, and \texttt{Bmatrix} for braces are all available.

\section{Units and Numerical Formatting}

Use the \texttt{siunitx} package for consistent typesetting of physical quantities and units.

\paragraph{Examples.}
\begin{verbatim}
\SI{3.6}{\meter\per\second}
\SI{9.81}{\meter\per\square\second}
\SIrange{0}{100}{\celsius}
\end{verbatim}

\texttt{siunitx} automatically applies proper spacing and typography between numbers and units.

\section{Theorem, Lemma, and Definition Environments}

Formal statements such as theorems, lemmas, and definitions can be declared using \texttt{amsthm}.  
\gls{novathesis} does not predefine them, allowing institutions to apply their own naming conventions.

\paragraph{Example Definition.}
Add in \texttt{0-Config/5\_packages.tex}:
\begin{verbatim}
\newtheorem{theorem}{Theorem}[chapter]
\newtheorem{lemma}[theorem]{Lemma}
\theoremstyle{definition}
\newtheorem{definition}[theorem]{Definition}
\end{verbatim}

Then, in the main text:
\begin{verbatim}
\begin{theorem}[Pythagoras]
For any right triangle, $a^2 + b^2 = c^2$.
\end{theorem}
\end{verbatim}

Each theorem will be numbered per chapter (e.g., Theorem~2.1).

\paragraph{Proof Environment.}
Use the standard \verb|\begin{proof}...\end{proof}| block, which automatically appends a Q.E.D. symbol.

\section{Algorithm Presentation}
\label{sec:algorithms}

Algorithms are formal representations of computational procedures or workflows.  
\gls{novathesis} provides two methods for including them:
\begin{itemize}
  \item As pseudocode environments using \texttt{algorithm} and \texttt{algpseudocode};
  \item As standard floats using \texttt{algorithmicx}.
\end{itemize}

\paragraph{Example.}
\begin{verbatim}
\begin{algorithm}[htbp]
\caption{Gradient Descent Optimization}
\label{alg:gradient}
\begin{algorithmic}[1]
  \State Initialize $\theta \leftarrow \theta_0$
  \For{$t = 1$ \textbf{to} $T$}
    \State Compute gradient $g_t = \nabla_\theta J(\theta_t)$
    \State Update parameters $\theta_{t+1} = \theta_t - \eta g_t$
  \EndFor
\end{algorithmic}
\end{algorithm}
\end{verbatim}

The \texttt{[1]} option enables line numbering for pseudocode clarity.  
Refer to the algorithm as:
\begin{verbatim}
As shown in Algorithm~\ref{alg:gradient}, parameters converge iteratively.
\end{verbatim}

\paragraph{Styling Algorithms.}
Adjust appearance in \texttt{0-Config/5\_packages.tex}:
\begin{verbatim}
\floatname{algorithm}{Algorithm}
\renewcommand{\listalgorithmname}{List of Algorithms}
\end{verbatim}

\section{Equations within Algorithms}

Mathematical expressions may be embedded directly within pseudocode lines:

\begin{verbatim}
\State $w_{i+1} = w_i - \eta \frac{\partial L}{\partial w_i}$
\end{verbatim}

Ensure consistent symbol usage between text, equations, and pseudocode for readability.

\section{Special Mathematical Symbols}

\texttt{amssymb} provides additional symbol libraries:
\begin{quote}
\verb|\mathbb{R}|, \verb|\mathbb{N}|, \verb|\forall|, \verb|\exists|, \verb|\nabla|, \verb|\partial|, \verb|\in|, \verb|\subset|.
\end{quote}

For bold vectors or matrices, use:
\begin{verbatim}
\bm{v}, \bm{A}
\end{verbatim}

Avoid using \verb|\mathbf| for Greek letters; use \texttt{bm} instead.

\section{Equation Spacing and Alignment Standards}

\gls{novathesis} automatically manages inter-equation spacing consistent with institutional standards.  
To adjust spacing manually:

\begin{verbatim}
\setlength{\abovedisplayskip}{12pt}
\setlength{\belowdisplayskip}{12pt}
\end{verbatim}

For alignment in theorems or proofs, use the \texttt{alignat} environment for precise column control.

\section{Mathematical Fonts and Typography}

The choice of math font family depends on the selected font theme (see Chapter~\ref{sec:new-font-theme}).  
When using XeLaTeX or LuaLaTeX, \texttt{unicode-math} can be loaded to access OpenType math fonts such as \texttt{Latin Modern Math} or \texttt{XITS Math}.

\paragraph{Example.}
\begin{verbatim}
\usepackage{unicode-math}
\setmathfont{XITS Math}
\end{verbatim}

This approach ensures typographic harmony between text and mathematical symbols.

\section{Multi-line Proofs and Derivations}

For formal derivations spanning multiple steps, use \texttt{align} with text annotations:

\begin{verbatim}
\begin{align}
f(x) &= (x + a)^2 \\
     &= x^2 + 2ax + a^2 \nonumber\\
     &\text{where $a$ is a constant.}
\end{align}
\end{verbatim}

Text inserted via \verb|\text| ensures consistent font and spacing.

\section{Notation Consistency}

\begin{itemize}
  \item Use italic letters for scalar variables (\(x, y, z\));
  \item Use bold lowercase letters for vectors (\(\bm{x}\));
  \item Use bold uppercase letters for matrices (\(\bm{A}\));
  \item Reserve calligraphic capitals (\(\mathcal{L}\)) for sets or functionals;
  \item Avoid mixing notations unless required by discipline-specific standards.
\end{itemize}

\section{Common Errors and Remedies}

\paragraph{Equation exceeds margin.}
Use \verb|\resizebox|, \verb|multline|, or break long expressions manually.

\paragraph{Undefined symbol.}
Ensure the relevant package (e.g., \texttt{amssymb}) is loaded.

\paragraph{Mismatched brackets.}
All math delimiters must be properly paired; use \verb|\left| and \verb|\right| for automatic sizing.

\paragraph{Incorrect alignment in systems.}
Use \texttt{align} instead of manual spacing with \verb|&|.

\section{Best Practices}

\begin{enumerate}
  \item Always label important equations for cross-referencing.
  \item Keep equations centered and consistent in font size.
  \item Define notation in the first chapter or a dedicated section.
  \item Avoid excessive inline formulas; use display equations for clarity.
  \item Align related equations using \texttt{align} or \texttt{aligned}.
  \item Maintain typographic uniformity across equations, tables, and text.
\end{enumerate}

\section{Summary}

\gls{novathesis} provides a comprehensive environment for mathematical typesetting and algorithm presentation.  
By leveraging \texttt{amsmath}, \texttt{amsthm}, and \texttt{algorithmicx}, authors can produce professional-quality mathematical content consistent with academic and institutional standards.  
Following the best practices and configuration techniques in this chapter ensures clarity, reproducibility, and typographic harmony across all formal and computational content.

