%!TEX root = ../template.tex
%%%%%%%%%%%%%%%%%%%%%%%%%%%%%%%%%%%%%%%%%%%%%%%%%%%%%%%%%%%%%%%%%%%%
%% chapter12.tex
%% NOVA thesis document file
%%
%% Chapter with lots of dummy text
%%%%%%%%%%%%%%%%%%%%%%%%%%%%%%%%%%%%%%%%%%%%%%%%%%%%%%%%%%%%%%%%%%%%

\typeout{NT FILE chapter12.tex}%

\chapter{Cross-Referencing, Indexes, and Glossaries}
\label{chap:crossref}

\section{Overview}

Accurate cross-referencing and terminology management are essential to the clarity and navigability of any academic thesis.  
\gls{novathesis} provides an integrated framework for managing internal references, hyperlinks, acronyms, glossaries, and indexes.  
This system is built upon standard \LaTeX{} packages—\texttt{hyperref}, \texttt{cleveref}, and \texttt{glossaries}—and is configured to operate automatically in both printed and digital versions of the document.

This chapter explains how to define and reference document elements, generate glossaries and acronym lists, and produce subject indexes.

\section{Internal Cross-References}

All document elements (chapters, sections, figures, tables, equations, algorithms, etc.) can be cross-referenced using \verb|\label| and \verb|\ref|.

\paragraph{Basic Usage.}
\begin{verbatim}
\section{Introduction}
\label{sec:intro}
...
As explained in Section~\ref{sec:intro}, the motivation for this study...
\end{verbatim}

Labels should follow a structured prefix convention:
\begin{quote}
\texttt{chap:}, \texttt{sec:}, \texttt{fig:}, \texttt{tab:}, \texttt{eq:}, \texttt{alg:}, etc.
\end{quote}

This naming scheme avoids conflicts and improves searchability.

\section{The \texttt{cleveref} Package}

\gls{novathesis} preloads \texttt{cleveref}, which automatically formats references with appropriate names and numbering.  
It also handles ranges and multiple references.

\paragraph{Example.}
\begin{verbatim}
As seen in \cref{fig:setup,tab:results}, the experimental conditions were identical.
\end{verbatim}

Output:
\begin{quote}
As seen in Figures~2.1 and~2.2, the experimental conditions were identical.
\end{quote}

\paragraph{Ranges.}
\begin{verbatim}
See \crefrange{eq:first}{eq:last} for the full derivation.
\end{verbatim}

\paragraph{Capitalization.}
Use \verb|\Cref| (capital “C”) at the start of a sentence:
\begin{verbatim}
\Cref{sec:methodology} provides the full description of the approach.
\end{verbatim}

\paragraph{Localization.}
Reference names (“Figure”, “Table”, “Equation”, etc.) are automatically translated according to the document language (see Chapter~\ref{chap:multilanguage}).

\section{Hyperlinks and PDF Metadata}

\gls{novathesis} loads the \texttt{hyperref} package with configuration consistent with institutional requirements.  
All cross-references, table-of-contents entries, and URLs become clickable in the generated PDF.

\paragraph{Default PDF Properties.}
The class automatically sets:
\begin{itemize}
  \item \texttt{pdfauthor}, \texttt{pdftitle}, \texttt{pdfsubject}, and \texttt{pdfkeywords};
  \item Link colors depending on the selected build mode (\texttt{media=screen} or \texttt{media=paper});
  \item Bookmarks and document structure for all major headings.
\end{itemize}

\paragraph{Media Modes.}
\begin{verbatim}
\ntsetup{media=screen} % colored links
\ntsetup{media=paper}  % black links for print
\end{verbatim}

\paragraph{Customizing Colors.}
If necessary, override in \texttt{0-Config/5\_packages.tex}:
\begin{verbatim}
\hypersetup{
  linkcolor=blue,
  citecolor=darkgray,
  urlcolor=navy
}
\end{verbatim}

\paragraph{Referencing URLs and DOIs.}
\begin{verbatim}
\url{https://novathesis.github.io}
\href{https://doi.org/10.1234/abcd}{DOI: 10.1234/abcd}
\end{verbatim}

\section{Acronyms and Abbreviations}

\gls{novathesis} integrates the \texttt{glossaries} package to manage acronyms and abbreviations.  
Acronyms are defined once and automatically expanded upon first use.

\paragraph{Defining Acronyms.}
Add entries in \texttt{1-FrontMatter/acronyms.tex}:
\begin{verbatim}
\newacronym{ml}{ML}{Machine Learning}
\newacronym{ai}{AI}{Artificial Intelligence}
\end{verbatim}

\paragraph{Using Acronyms.}
\begin{verbatim}
\gls{ml} models are a subset of \gls{ai} systems.
\end{verbatim}

The first occurrence prints the full form followed by the abbreviation:  
“Machine Learning (ML) models are a subset of Artificial Intelligence (AI) systems.”  
Subsequent uses display only the short form.

\paragraph{Plural and Capitalized Forms.}
\begin{verbatim}
\glspl{ml}      % plural form
\Gls{ai}        % capitalized form
\Glspl{ai}      % capitalized plural
\end{verbatim}

\paragraph{List of Acronyms.}
Enable in \texttt{0-Config/6\_list\_of.tex}:
\begin{verbatim}
\ntaddlistof{acronyms}
\end{verbatim}

This list will appear in the front matter.

\section{Glossaries}

Glossaries are used to define technical terms or symbols.  
Each entry includes a key, a name, and a description.

\paragraph{Example.}
Add definitions to \texttt{1-FrontMatter/glossary.tex}:
\begin{verbatim}
\newglossaryentry{entropy}{
  name=Entropy,
  description={A measure of disorder or randomness in a system}
}
\newglossaryentry{overfitting}{
  name=Overfitting,
  description={Modeling noise instead of the underlying relationship}
}
\end{verbatim}

Use terms in the text:
\begin{verbatim}
\gls{entropy} is a fundamental concept in thermodynamics.
\end{verbatim}

\paragraph{Printing the Glossary.}
At the end of the document, add:
\begin{verbatim}
\printglossaries
\end{verbatim}

or enable the automatic inclusion in \texttt{0-Config/6\_list\_of.tex}:
\begin{verbatim}
\ntaddlistof{glossaries}
\end{verbatim}

\section{Generating Glossaries}

When compiling locally, the Makefile automatically invokes the \texttt{makeglossaries} command.  
If run manually:

\begin{verbatim}
makeglossaries template
latexmk -xelatex template.tex
\end{verbatim}

Ensure that the glossary definitions are placed in separate files (e.g., \texttt{acronyms.tex}, \texttt{glossary.tex}) and that they are included through \texttt{0-Config/4\_files.tex}.

\paragraph{Overleaf Limitation.}
Overleaf’s free tier does not support the \texttt{makeglossaries} executable.  
Either upgrade to a paid account or generate the glossary locally and upload the resulting \texttt{.gls} and \texttt{.glo} files manually.

\section{Multiple Glossaries}

The \texttt{glossaries} package supports multiple glossary types (e.g., symbols, nomenclature, abbreviations).  
Declare new types in \texttt{0-Config/5\_packages.tex}:

\begin{verbatim}
\newglossary[symg]{symbols}{sym}{sbl}{List of Symbols}
\end{verbatim}

Then define entries:
\begin{verbatim}
\newglossaryentry{alpha}{
  type=symbols,
  name={\ensuremath{\alpha}},
  description={Learning rate in optimization algorithms}
}
\end{verbatim}

\paragraph{Printing.}
\begin{verbatim}
\printglossary[type=symbols]
\end{verbatim}

\section{Indexes}
\label{sec:indexes}

An index provides a comprehensive alphabetical list of topics, authors, or keywords referenced throughout the document.  
\gls{novathesis} integrates the standard \texttt{makeidx} and \texttt{imakeidx} packages.

\paragraph{Enabling Indexing.}
In \texttt{0-Config/5\_packages.tex}, uncomment:
\begin{verbatim}
\makeindex
\end{verbatim}

\paragraph{Creating Entries.}
Mark terms for indexing using:
\begin{verbatim}
\index{machine learning}
\index{artificial intelligence!supervised learning}
\end{verbatim}

Nested entries use the “!” separator.

\paragraph{Printing the Index.}
Add the following at the end of the document or enable it in \texttt{0-Config/6\_list\_of.tex}:
\begin{verbatim}
\printindex
\end{verbatim}

\paragraph{Building the Index.}
If not managed automatically, run:
\begin{verbatim}
makeindex template
latexmk -xelatex template.tex
\end{verbatim}

\section{Nomenclature and Symbols}

For technical documents requiring a list of symbols or nomenclature, reuse the glossary mechanism or employ the \texttt{nomencl} package.

\paragraph{Example with \texttt{nomencl}.}
\begin{verbatim}
\usepackage{nomencl}
\makenomenclature

\nomenclature{$E$}{Energy (J)}
\nomenclature{$m$}{Mass (kg)}
\nomenclature{$c$}{Speed of light (m/s)}

\printnomenclature
\end{verbatim}

Compile with:
\begin{verbatim}
makeindex template.nlo -s nomencl.ist -o template.nls
\end{verbatim}

\section{Cross-Referencing Best Practices}

\begin{itemize}
  \item Always assign meaningful label names reflecting the element type;
  \item Place \verb|\label| immediately after the \verb|\caption| command in floats;
  \item Use \verb|\cref|/\verb|\Cref| instead of manual “Figure~\verb|\ref|{}” combinations;
  \item Verify that every reference resolves after two compilation passes;
  \item Avoid duplicating labels across different environments;
  \item For equations, use short and descriptive labels (e.g., \texttt{eq:navier-stokes});
  \item Regularly clean auxiliary files (\texttt{make clean}) to remove stale references.
\end{itemize}

\section{Common Issues and Remedies}

\paragraph{“Label(s) may have changed.”}  
This message is informational; recompile once more to synchronize references.

\paragraph{“Undefined references” or “???” in output.}  
Ensure that \verb|\label| names match their \verb|\ref| usage exactly (case-sensitive).

\paragraph{Glossary not generated.}  
Run \texttt{makeglossaries template} manually, then recompile.

\paragraph{Acronym list empty.}  
Ensure that acronyms are defined before first use and that the list is included in \texttt{0-Config/6\_list\_of.tex}.

\paragraph{Broken hyperlinks in Overleaf.}  
Recompile using XeLaTeX or LuaLaTeX; ensure that \texttt{hyperref} is not redefined by another package.

\section{Summary}

\gls{novathesis} integrates comprehensive cross-referencing, hyperlinking, glossary, and index systems that operate seamlessly across languages and compilation modes.  
By defining clear labels, consistent acronyms, and structured glossaries, authors ensure that readers can navigate the document effectively and that institutional standards for cross-referencing and terminology management are satisfied.  
All such mechanisms function automatically within the \gls{novathesis} workflow, requiring only minimal configuration.
