%!TEX root = ../template.tex
%%%%%%%%%%%%%%%%%%%%%%%%%%%%%%%%%%%%%%%%%%%%%%%%%%%%%%%%%%%%%%%%%%%%
%% chapter8.tex
%% NOVA thesis document file
%%
%% Chapter with lots of dummy text
%%%%%%%%%%%%%%%%%%%%%%%%%%%%%%%%%%%%%%%%%%%%%%%%%%%%%%%%%%%%%%%%%%%%

\typeout{NT FILE chapter8.tex}%

\chapter{Multi-language Support}
\label{chap:multilanguage}

\section{Overview}

\gls{novathesis} provides comprehensive multilingual capabilities, enabling users to produce theses in one or more languages with full localization of structural elements such as titles, captions, and lists.  
Language management is built upon the standard \texttt{babel} and \texttt{polyglossia} frameworks, depending on the selected engine, and integrates seamlessly with the class configuration system.

This chapter describes how to configure document languages, define multilingual abstracts, manage localized metadata, and ensure consistent formatting across all language sections.  
It also outlines advanced techniques for bilingual and multilingual theses.

\section{Supported Languages}

\gls{novathesis} includes predefined string sets for the following languages:

\begin{center}
\texttt{EN} (English),  
\texttt{PT} (Portuguese),  
\texttt{ES} (Spanish),  
\texttt{FR} (French),  
\texttt{DE} (German),  
\texttt{IT} (Italian),  
\texttt{GR} (Greek),  
\texttt{UK} (Ukrainian).
\end{center}

Each language provides localized labels for chapter titles, lists of figures and tables, appendices, and cover-page elements.  
Additional languages can be defined by creating a new file in \texttt{\gls{novathesis}Files/Strings/} (see Chapter~\ref{sec:new-language}).

\section{Language Architecture}

\gls{novathesis} distinguishes between three levels of language configuration:

\begin{description}
  \item[\textbf{Main Language}] — Governs the language of the entire document, including headings, captions, and the main body text.
  \item[\textbf{Abstract Languages}] — Defines the set of languages in which abstracts are provided.
  \item[\textbf{Extra Languages}] — Enables occasional translations, quotations, or bilingual text fragments within the document.
\end{description}

All language behavior is controlled by the \verb|\ntsetup| keys in \texttt{0-Config/1\_novathesis.tex}.

\section{Setting the Main Language}
\label{sec:main-language}

The main document language is selected using the \texttt{mainlanguage} key:

\begin{verbatim}
\ntsetup{mainlanguage=pt}
\end{verbatim}

Accepted values correspond to the two-letter ISO codes listed above.  
This determines:
\begin{itemize}
  \item The default hyphenation and typographic conventions;
  \item Localization of structural strings (chapter, figure, table, appendix, etc.);
  \item The order and naming of automatically generated lists and tables of contents;
  \item The primary language for front-matter metadata and cover elements.
\end{itemize}

\paragraph{Example: English Thesis}
\begin{verbatim}
\ntsetup{mainlanguage=en}
\end{verbatim}
The entire document, including chapter headings and captions, will appear in English.

\section{Adding Secondary and Abstract Languages}
\label{sec:abstract-languages}

To include additional languages (typically for abstracts), declare them as a comma-separated list in the \texttt{abstractorder} key:

\begin{verbatim}
\ntsetup{abstractorder={pt,en}}
\end{verbatim}

This example generates two abstracts: first in Portuguese, then in English.  
Each abstract must have a corresponding file declared in \texttt{0-Config/4\_files.tex}:

\begin{verbatim}
% 0-Config/4_files.tex
\ntaddfile{abstract}[pt]{abstract-pt}
\ntaddfile{abstract}[en]{abstract-en}
\end{verbatim}

\paragraph{Automatic Localization.}
Each abstract automatically uses the correct localized title (“Resumo”, “Abstract”, etc.) and follows the typographic conventions of its respective language.

\paragraph{Example: Bilingual Abstract Files.}

\begin{verbatim}
% 1-FrontMatter/abstract-pt.tex
\begin{abstract}
  Este trabalho analisa a aplicação de algoritmos de aprendizagem automática
  em séries temporais financeiras. Foram exploradas técnicas supervisionadas
  e não supervisionadas com foco na previsibilidade e estabilidade.
\end{abstract}

% 1-FrontMatter/abstract-en.tex
\begin{abstract}
  This work analyzes the application of machine learning algorithms
  to financial time-series data. Both supervised and unsupervised
  approaches were explored, emphasizing predictability and stability.
\end{abstract}
\end{verbatim}

\section{Localized Metadata}
\label{sec:localized-metadata}

\gls{novathesis} allows most metadata fields to be defined in multiple languages using the same \verb|\ntsetup| interface.  
The class automatically selects the version corresponding to the current language context.

\paragraph{Example: Thesis Title and Keywords.}
\begin{verbatim}
\nttitle(main,pt){Análise de Dados Climáticos}
\nttitle(main,en){Climate Data Analysis}

\ntkeywords(pt){Meteorologia, Modelação, Previsão}
\ntkeywords(en){Meteorology, Modelling, Forecasting}
\end{verbatim}

When generating the front matter and covers, \gls{novathesis} will automatically display the correct titles and keywords for each language context.

\paragraph{Optional subtitles.}
Subtitles can also be defined per language:
\begin{verbatim}
\nttitle(sub,pt){Modelos Estatísticos e Aprendizagem Automática}
\nttitle(sub,en){Statistical Models and Machine Learning}
\end{verbatim}

\section{Changing Language within the Document}

Occasionally, users may need to insert text or quotations in a language other than the main one.  
The command \verb|\foreignlanguage| may be used directly:

\begin{verbatim}
\foreignlanguage{pt}{A aprendizagem automática é uma área em rápido crescimento.}
\end{verbatim}

This ensures correct hyphenation and punctuation rules for the inserted text.  
In XeLaTeX and LuaLaTeX, \gls{novathesis} automatically selects the appropriate font encodings through \texttt{fontspec}.

\section{Multilingual Captions and Cross-References}

Captions for figures and tables follow the current language context automatically.  
When working in a mixed-language document, use \verb|\selectlanguage{<code>}| to enforce specific localization before a float environment.

\begin{verbatim}
\selectlanguage{en}
\begin{figure}
  \centering
  \includegraphics[width=0.7\linewidth]{5-Figures/sample}
  \caption{Example of a multilingual caption.}
\end{figure}

\selectlanguage{pt}
\begin{figure}
  \centering
  \includegraphics[width=0.7\linewidth]{5-Figures/sample}
  \caption{Exemplo de legenda multilíngue.}
\end{figure}
\end{verbatim}

\section{Bilingual Theses}
\label{sec:bilingual-thesis}

Some institutions require bilingual theses, where the abstract, keywords, and sometimes chapter titles appear in two languages.  
\gls{novathesis} supports this through the coordinated use of the \texttt{mainlanguage} and \texttt{abstractorder} keys.

\subsection{Recommended Configuration}

\begin{verbatim}
\ntsetup{
  mainlanguage=en,
  abstractorder={pt,en},
  lang/extra={pt}
}
\end{verbatim}

This configuration produces an English thesis with Portuguese as a secondary language for abstracts and optional quotations.

\subsection{Localized Lists and Captions}

All “List of …” elements (Figures, Tables, Glossaries) automatically use the language of the main document.  
To provide translated captions or headings, redefine only the necessary strings via \verb|\ntlangsetup|:

\begin{verbatim}
\ntlangsetup{pt/listfigurename=Lista de Figuras}
\ntlangsetup{pt/listtablename=Lista de Tabelas}
\end{verbatim}

\section{Defining a New Language}
\label{sec:defining-new-language}

If your thesis requires a language not shipped with \gls{novathesis}, follow these steps:

\begin{enumerate}
  \item Create a new file in \texttt{\gls{novathesis}Files/Strings/}, named after the language code, e.g.,
  \begin{verbatim}
  \gls{novathesis}Files/Strings/sv.tex
  \end{verbatim}
  for Swedish.
  \item Copy the structure of an existing file (e.g., \texttt{en.tex}) and translate all values:
  \begin{verbatim}
  \ntlangsetup{
    sv/chaptername=Kapitel,
    sv/contentsname=Innehåll,
    sv/figuresname=Figurer,
    sv/tablesname=Tabeller,
    sv/appendixname=Bilaga
  }
  \end{verbatim}
  \item Update \texttt{1\_novathesis.tex}:
  \begin{verbatim}
  \ntsetup{mainlanguage=sv}
  \end{verbatim}
  \item Compile using XeLaTeX or LuaLaTeX (required for Unicode languages).
\end{enumerate}

\section{Right-to-Left (RTL) Languages}

For languages such as Arabic or Hebrew, XeLaTeX and LuaLaTeX provide native right-to-left text rendering through \texttt{polyglossia}.  
\gls{novathesis} detects and configures \texttt{polyglossia} automatically when an RTL language is selected as the main language.

\paragraph{Example:}
\begin{verbatim}
\ntsetup{mainlanguage=he}
\end{verbatim}

Ensure that the fonts chosen in \texttt{1\_novathesis.tex} support the corresponding script, for example:

\begin{verbatim}
\setmainfont{Amiri}
\setsansfont{Scheherazade New}
\end{verbatim}

\section{Multilingual Bibliography}

The bibliography system supports multilingual entries through \texttt{biblatex}.  
Each entry can include language tags:
\begin{verbatim}
@book{example2024,
  author    = {M. Silva},
  title     = {Introdução à Modelação},
  year      = {2024},
  language  = {portuguese}
}
\end{verbatim}

\paragraph{Automatic Localization.}
When the \texttt{babel} package is active, bibliography headings and reference labels are automatically translated according to the main document language.

To print the bibliography in multiple languages, load \texttt{biblatex} with the \texttt{autolang=other} option in \texttt{2\_biblatex.tex}:
\begin{verbatim}
\ntbibsetup{
  backend=biber,
  autolang=other
}
\end{verbatim}

\section{Best Practices for Multilingual Documents}

\begin{enumerate}
  \item Define the main language first; add secondary ones only as required.
  \item Maintain one abstract file per language, clearly named.
  \item Do not mix multiple languages in a single paragraph; switch context explicitly.
  \item Use XeLaTeX or LuaLaTeX for Unicode and OpenType font compatibility.
  \item Verify that hyphenation patterns and quotation marks are correct in each language.
  \item Ensure that all metadata fields (titles, keywords, degree name) have corresponding translations.
  \item Check the cover page output for correct accent rendering.
\end{enumerate}

\section{Common Issues and Solutions}

\paragraph{Accented characters not displaying.}
Ensure all files are UTF-8 encoded and that the selected fonts support the necessary characters.

\paragraph{Wrong language on cover or abstract.}
Verify that only one instance of each \verb|\nttitle(main,<lang>)| or \verb|\ntdegreename(<lang>)| pair is active.  
If multiple definitions are provided for the same language, the last one will take precedence.

\paragraph{Hyphenation errors or mixed punctuation.}
Confirm that the correct main language is set in \texttt{1\_novathesis.tex} and that the proper language package (\texttt{babel} or \texttt{polyglossia}) is loaded automatically by the class.

\section{Summary}

\gls{novathesis} provides robust, flexible multilingual capabilities that cover every aspect of academic document preparation—from abstracts and metadata to captions, covers, and bibliography.  
By combining institutional presets with standard \LaTeX{} language mechanisms, users can prepare bilingual or multilingual theses effortlessly, without modifying the core class.  
Adhering to the configuration principles described in this chapter ensures linguistic accuracy, typographic consistency, and full compliance with institutional submission standards.
