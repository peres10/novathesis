%!TEX root = ../template.tex
%%%%%%%%%%%%%%%%%%%%%%%%%%%%%%%%%%%%%%%%%%%%%%%%%%%%%%%%%%%%%%%%%%%%
%% chapter13.tex
%% NOVA thesis document file
%%
%% Chapter with lots of dummy text
%%%%%%%%%%%%%%%%%%%%%%%%%%%%%%%%%%%%%%%%%%%%%%%%%%%%%%%%%%%%%%%%%%%%

\typeout{NT FILE chapter13.tex}%

\chapter{Printing, Submission, and Archiving}
\label{chap:printing}

\section{Overview}

The final stage of thesis preparation involves generating the definitive document for submission, publication, and long-term preservation.  
\gls{novathesis} provides explicit mechanisms for producing print-ready and digital versions, ensuring typographic consistency, metadata accuracy, and compliance with institutional and archival standards.

This chapter explains how to configure and verify the output mode, prepare the thesis for submission, and archive all necessary files for institutional repositories and digital libraries.

\section{Output Modes and Media Configuration}

\gls{novathesis} distinguishes between several build and media modes that govern hyperlink appearance, color usage, watermarking, and inclusion of draft marks.  
These modes are defined in the configuration file \texttt{0-Config/1\_novathesis.tex}.

\paragraph{Example.}
\begin{verbatim}
\ntsetup{
  mode=final,
  media=screen
}
\end{verbatim}

\begin{description}
  \item[\texttt{mode}] — controls internal settings such as watermark visibility and compilation strictness:
    \begin{itemize}
      \item \texttt{working} — for drafting, includes TODO notes and colored boxes;
      \item \texttt{provisional} — for near-final review, without watermark but with colored links;
      \item \texttt{final} — for official submission; disables all draft features.
    \end{itemize}
  \item[\texttt{media}] — determines color profile and link appearance:
    \begin{itemize}
      \item \texttt{screen} — colored hyperlinks, optimized for digital reading;
      \item \texttt{paper} — black hyperlinks, optimized for grayscale printing.
    \end{itemize}
\end{description}

\paragraph{Command-Line Override.}
Modes may also be set during compilation:
\begin{verbatim}
make MODE=final MEDIA=paper xe
\end{verbatim}

\section{Verifying the Final Document}

Before printing or submission, confirm the following:

\begin{enumerate}
  \item All front matter elements (title page, abstracts, acknowledgments) are present and updated;
  \item No visible TODO notes or draft indicators remain;
  \item Hyperlinks are functional and appear as black (for print) or colored (for digital);
  \item Table of contents, lists of figures/tables, and bibliography are complete;
  \item Fonts are embedded, and the PDF complies with institutional requirements.
\end{enumerate}

Run a full clean build to remove temporary files and regenerate all references:
\begin{verbatim}
make clean-all
make xe
\end{verbatim}

\section{Printing Configuration}

\paragraph{Paper Size and Margins.}
\gls{novathesis} defaults to A4 paper (\SI{210}{\milli\meter} × \SI{297}{\milli\meter}), the standard for European academic submissions.  
Margins are defined in \texttt{0-Config/0\_memoir.tex} and follow institutional standards (usually 2.5~cm inner, 3.0~cm outer, 2.5~cm top and bottom).  
To adjust:

\begin{verbatim}
\settrimmedsize{297mm}{210mm}{*}
\settrims{0pt}{0pt}
\settypeblocksize{220mm}{145mm}{*}
\setlrmargins{*}{*}{1.2}
\setulmargins{*}{*}{1.2}
\checkandfixthelayout
\end{verbatim}

\paragraph{Color Policy.}
When printing, use the \texttt{media=paper} option to ensure all text, hyperlinks, and references are rendered in black and gray scale.  
Institutional print shops may require 100\% black output for certain submissions.

\paragraph{Duplex Printing.}
The template supports duplex (double-sided) printing by default, with mirrored margins.  
For single-sided printing, disable in \texttt{0-Config/1\_novathesis.tex}:

\begin{verbatim}
\ntsetup{twoside=false}
\end{verbatim}

\section{Digital Submission}

\paragraph{File Format.}
All final submissions must be delivered as PDF documents.  
\gls{novathesis} automatically generates compliant PDFs with embedded fonts and searchable text.

\paragraph{File Naming.}
Use the institutional naming convention, typically:
\begin{verbatim}
Lastname_Firstname_Thesis_YEAR.pdf
\end{verbatim}

Avoid spaces, special characters, or diacritics in filenames.

\paragraph{Cover and Metadata.}
Ensure that all metadata fields (author, title, degree, supervisor, and date) are defined in \texttt{1\_novathesis.tex}.  
These populate both the title page and the embedded PDF metadata via \texttt{hyperref}.

\paragraph{Example.}
\begin{verbatim}
\ntauthor{Maria Silva}
\nttitle(main,en){Analysis of Renewable Energy Markets}
\ntdegree{Doctor of Philosophy in Economics}
\ntsupervisor{Prof.~João Pereira}
\ntdate{March 2025}
\end{verbatim}

\section{PDF/A Compliance for Archiving}

Many repositories require archival formats such as PDF/A-1b or PDF/A-2u, ensuring long-term preservation and font integrity.

\paragraph{Validation.}
After compiling, verify compliance using:
\begin{verbatim}
pdfinfo -meta template.pdf
\end{verbatim}

or with specialized tools such as \emph{veraPDF}.

\paragraph{Creating PDF/A Files.}
\gls{novathesis} does not automatically enforce PDF/A mode, but compliance can be achieved using \texttt{ghostscript}:

\begin{verbatim}
gs -dPDFA=2 -dBATCH -dNOPAUSE -dNOOUTERSAVE \
   -sProcessColorModel=DeviceCMYK \
   -sDEVICE=pdfwrite \
   -sPDFACompatibilityPolicy=1 \
   -sOutputFile=Thesis_PDFA.pdf template.pdf
\end{verbatim}

\paragraph{Font Embedding.}
Ensure that all fonts are embedded; this is automatically managed when using TeX~Live or MacTeX with \texttt{fontspec} under XeLaTeX or LuaLaTeX.

\section{Institutional Packaging and Submission}

Institutions typically require submission of the following files:

\begin{itemize}
  \item Final thesis PDF (compiled in \texttt{mode=final});
  \item Source code archive (optional, but recommended for reproducibility);
  \item Supplementary data (datasets, scripts, figures);
  \item Licensing and originality statements, if applicable.
\end{itemize}

\paragraph{Automated Packaging.}
\gls{novathesis} includes a Makefile target that packages the complete project for submission:

\begin{verbatim}
make package
\end{verbatim}

This generates an archive:
\begin{verbatim}
ThesisPackage_<author>_<date>.zip
\end{verbatim}

containing:
\begin{itemize}
  \item \texttt{template.pdf} (final output);
  \item \texttt{0-Config/}, \texttt{1-FrontMatter/}, \texttt{2-MainMatter/}, \texttt{3-BackMatter/};
  \item \texttt{4-Bibliography/} and \texttt{5-Figures/};
  \item The Makefile and configuration files;
  \item A \texttt{README.txt} with metadata summary.
\end{itemize}

\paragraph{Optional Source Exclusion.}
To exclude the internal class files:
\begin{verbatim}
make package SRC=false
\end{verbatim}

\section{Digital Repository Submission}

Many universities require electronic deposit into an institutional repository or national archive.  
Before uploading:

\begin{itemize}
  \item Confirm that the PDF opens without warnings in Acrobat Reader;
  \item Verify embedded metadata (author, title, subject, keywords);
  \item Check that page numbering and bookmarks are correct;
  \item Include separate files for appendices or supplementary material if requested.
\end{itemize}

\paragraph{Embargo and Licensing.}
For works under embargo or with restricted data:
\begin{itemize}
  \item Include the embargo statement in both the abstract and metadata;
  \item Specify Creative Commons or institutional licensing in the submission form;
  \item Retain a signed copy of the copyright declaration.
\end{itemize}

\section{Archiving the Project}

To ensure long-term reproducibility, the complete project directory should be archived in a versioned repository or institutional backup system.  
Include:

\begin{itemize}
  \item The \LaTeX{} source files and configuration directories;
  \item The final compiled PDF;
  \item A README documenting the compilation environment (engine, TeX distribution, version);
  \item The bibliography files and all figures;
  \item Checksums (e.g., SHA-256) for verification.
\end{itemize}

\paragraph{Example.}
\begin{verbatim}
sha256sum template.pdf > thesis_checksum.txt
\end{verbatim}

Store both the source and the checksum in the repository archive.

\section{Long-Term Preservation Best Practices}

\begin{itemize}
  \item Use open-source fonts (e.g., Latin Modern, XITS) for guaranteed embedding;
  \item Maintain a copy of the corresponding TeX~Live installer image;
  \item Include the \gls{novathesis} version number in your repository metadata;
  \item Prefer vector images (PDF, EPS) for figures to ensure scalability;
  \item Avoid proprietary or compressed font formats;
  \item Document all system dependencies (Biber, Ghostscript, etc.).
\end{itemize}

\paragraph{Version Tagging.}
The project version is recorded automatically in the class header:
\begin{verbatim}
\ProvidesClass{novathesis}[2025/04/07 v7.6.0 NOVA Thesis Template]
\end{verbatim}

Include this line in your submission documentation for traceability.

\section{Quality Assurance Checklist}

Before final submission, confirm the following items:

\begin{itemize}
  \item [ ] The document compiles in \texttt{mode=final} without errors or warnings;
  \item [ ] All figures and tables are properly numbered and referenced;
  \item [ ] All cross-references are resolved;
  \item [ ] The bibliography is complete and formatted consistently;
  \item [ ] Acronyms and glossaries are printed;
  \item [ ] The title page and metadata match institutional templates;
  \item [ ] Fonts are embedded, and the PDF opens cleanly on multiple platforms;
  \item [ ] A copy of the project has been archived.
\end{itemize}

\section{Sample Build Commands for Final Output}

\paragraph{Final Digital Version (screen).}
\begin{verbatim}
make MODE=final MEDIA=screen xe
\end{verbatim}

\paragraph{Print Version (paper).}
\begin{verbatim}
make MODE=final MEDIA=paper pdf
\end{verbatim}

\paragraph{Archive Packaging.}
\begin{verbatim}
make clean-all
make package
\end{verbatim}

\section{Summary}

This chapter consolidates all procedures required for producing, validating, and preserving the final version of a thesis prepared with \gls{novathesis}.  
By selecting the correct build mode, embedding complete metadata, and packaging the project according to institutional requirements, authors ensure that their work is reproducible, portable, and compliant with long-term archiving standards such as PDF/A.  
The result is a thesis that remains accessible and verifiable long after its original submission.
