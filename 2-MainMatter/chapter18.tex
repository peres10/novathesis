%!TEX root = ../template.tex

\chapter*{Afterword}
\addcontentsline{toc}{chapter}{Afterword}

\section*{Reflection}

The \gls{novathesis} Template was created to unify, simplify, and professionalize the process of academic thesis preparation at NOVA University Lisbon.  
Over time, it has become more than a formatting tool—it has grown into an open, community-driven framework that embodies academic rigor, typographic excellence, and technological transparency.

This manual concludes a comprehensive effort to document not only the technical structure of the template, but also the principles that sustain it:  
clarity, reproducibility, and collaboration.  
Every command, file, and configuration key described herein was designed to serve a single goal—allowing authors to focus on the quality of their research rather than the mechanics of presentation.

\section*{Sustainability and Community}

The longevity of \gls{novathesis} depends on the participation and curiosity of its users.  
Students, supervisors, and institutions are encouraged to:
\begin{itemize}
  \item Share improvements and report inconsistencies through the official repository;
  \item Translate or adapt the template for new languages or disciplines;
  \item Contribute documentation, presets, or design refinements;
  \item Promote open standards and reproducible research practices.
\end{itemize}

The maintainers believe that academic templates are living documents that evolve with the institutions they serve.  
Each contribution—no matter how small—preserves the continuity of a tradition rooted in precision, accessibility, and open knowledge.

\section*{Gratitude}

The \gls{novathesis} Coordination Team extends its sincere gratitude to all users, contributors, and institutions who have supported the project over the years.  
Your feedback, testing, and collaboration have made this template what it is today: a shared academic infrastructure built on trust and open science.

Special acknowledgment is given to the \LaTeX{} community at large—whose commitment to excellence, open licensing, and documentation has made professional-quality typesetting universally accessible.

\section*{Looking Forward}

The future of \gls{novathesis} lies in adaptability.  
As academic publishing evolves toward digital, accessible, and data-rich environments, the template will continue to expand—embracing new technologies while remaining faithful to its original purpose.  
Its codebase and documentation will persist as an open educational resource for future generations of researchers.

\section*{Final Words}

A thesis is both an individual and institutional achievement.  
\gls{novathesis} exists to support that process—ensuring that every thesis, regardless of discipline or language, meets the highest standards of clarity, consistency, and permanence.

\bigskip
\noindent\textit{Lisbon, April~2025}\\
\textit{The \gls{novathesis} Coordination Team}\\
\textit{NOVA University Lisbon}
