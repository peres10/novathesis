%!TEX root = ../template.tex
%%%%%%%%%%%%%%%%%%%%%%%%%%%%%%%%%%%%%%%%%%%%%%%%%%%%%%%%%%%%%%%%%%%
%% chapter1.tex
%% NOVA thesis document file
%%
%% Chapter with introduction
%%%%%%%%%%%%%%%%%%%%%%%%%%%%%%%%%%%%%%%%%%%%%%%%%%%%%%%%%%%%%%%%%%%

\typeout{NT FILE chapter1.tex}%

\chapter*{Preface}
\addcontentsline{toc}{chapter}{Preface}

\section*{Purpose of this Manual}

This manual is the official user guide for the \textbf{\gls{novathesis} Template}, a \LaTeX{}-based document class and project structure designed for the preparation of academic theses and dissertations.  
Its objective is to provide a complete, reproducible reference for students, supervisors, and institutions using the template for scientific and academic writing.

The manual covers the entire workflow, from installation and configuration to advanced customization, build automation, and troubleshooting.  
It also serves as a technical reference for system administrators who maintain institutional deployments or verify compliance with formatting policies.

All examples and configuration fragments shown in this document are directly compatible with the release version distributed in the \texttt{template-7.6.0} package.

\section*{Intended Audience}

The primary audience of this manual includes:
\begin{itemize}
  \item \textbf{Undergraduate, Master’s, and Doctoral students} preparing theses or dissertations using \LaTeX{};
  \item \textbf{Supervisors and examiners} verifying structural and typographic compliance;
  \item \textbf{System administrators and technical staff} responsible for configuring \LaTeX{} environments in institutional servers or laboratories;
  \item \textbf{Template maintainers and developers} extending or adapting the \gls{novathesis} class to new institutions.
\end{itemize}

The manual assumes basic familiarity with \LaTeX{} compilation and directory structures, but no prior experience with class development.

\section*{Scope}

This document describes:
\begin{itemize}
  \item The structure and configuration system of the \gls{novathesis} template;
  \item The supported installation methods (local and cloud-based);
  \item The automated build and packaging workflow using \texttt{make} and \texttt{latexmk};
  \item The customization mechanisms for institutions, covers, fonts, and metadata;
  \item Troubleshooting and diagnostic procedures for all major environments.
\end{itemize}

It does not provide a tutorial on general \LaTeX{} usage; readers seeking an introduction to \LaTeX{} syntax and document preparation should consult the standard documentation distributed with TeX~Live or available at \url{https://www.latex-project.org}.

\section*{Acknowledgments}

The \gls{novathesis} project was originally developed to standardize thesis formatting across the NOVA University Lisbon schools.  
It has since evolved into a modular and extensible platform for academic document preparation, thanks to contributions from faculty, researchers, and students.

The maintainers acknowledge the foundational work of the \texttt{memoir} class authors, as well as the developers of \texttt{biblatex}, \texttt{fontspec}, and the \LaTeX3 project team.  
Their tools provide the robust foundation upon which \gls{novathesis} is built.

Contributors to the template and documentation are listed in the project repository’s history.

\section*{Version and Maintenance Policy}

This manual corresponds to \gls{novathesis} version \textbf{7.6.0} (April~2025).  
Future releases will include incremental updates to both the class and this manual.  
Backward compatibility of configuration files is preserved whenever possible; any breaking changes will be explicitly documented in release notes.

Users are encouraged to verify the version printed on the cover page of the manual and in the class header:

\begin{verbatim}
\ProvidesClass{novathesis}[2025/04/07 v7.6.0 NOVA Thesis Template]
\end{verbatim}

\section*{Feedback and Support}

Questions, error reports, and feature requests may be submitted through the official project repository.  
When reporting issues, include:
\begin{itemize}
  \item The \gls{novathesis} version and date;
  \item The operating system and TeX distribution used;
  \item The compilation engine (\texttt{pdfLaTeX}, \texttt{XeLaTeX}, or \texttt{LuaLaTeX});
  \item A minimal working example reproducing the problem.
\end{itemize}

\section*{Citation}

When referencing this work in publications or institutional documentation, cite it as:

\begin{quote}
NOVA University Lisbon. \emph{\gls{novathesis} Template and User Manual}. Version~7.6.0, April~2025.  
Available at \url{https://github.com/novathesis/novathesis}.
\end{quote}

\section*{License}

The \gls{novathesis} Template and this manual are distributed under the terms of the \textbf{LaTeX Project Public License (LPPL)}, version~1.3c or later.  
You are free to use, distribute, and modify the template under the following conditions:
\begin{itemize}
  \item Any distributed modification must clearly identify the modified version;
  \item The original author and license notices must be preserved;
  \item Modified versions must not be distributed under the same file names as the original;
  \item The template and manual are provided \emph{without warranty}, express or implied.
\end{itemize}

A copy of the LPPL is included in the distribution as \texttt{LICENSE.txt} and available at \url{https://www.latex-project.org/lppl.txt}.

\section*{Disclaimer}

The \gls{novathesis} maintainers and affiliated institutions make no warranty regarding compliance with specific institutional submission requirements.  
Users are responsible for verifying that their final thesis or dissertation meets the formatting standards of their respective academic unit.

\section*{Revision History}

\begin{description}
  \item[Version 7.6.0] (April~2025) — Updated configuration architecture, revised institutional presets, and full manual rewrite.
  \item[Version 7.5.0] (September~2024) — Introduced modular font styles and extended build automation.
  \item[Version 7.4.0] (May~2024) — Added multi-institutional support and Overleaf compatibility improvements.
  \item[Version 7.0.0] (April~2023) — Major redesign introducing unified configuration and template structure.
\end{description}

\section*{Structure of this Document}

This manual is divided into the following main chapters:

\begin{enumerate}
  \item \textbf{Introduction} — Purpose and scope of the template;
  \item \textbf{Installation} — Local and cloud installation methods;
  \item \textbf{Document Structure} — File hierarchy and logical organization;
  \item \textbf{Makefile and Build System} — Automation and reproducible compilation;
  \item \textbf{Advanced Customization} — Extending and adapting the template;
  \item \textbf{Troubleshooting and FAQ} — Problem-solving and diagnostics;
  \item \textbf{Build Environment Reference} — Technical appendix for maintainers.
\end{enumerate}

Each chapter can be read independently, though readers new to \LaTeX{} should begin with Chapters~1 through~3 before attempting advanced customization.

\bigskip
\noindent\textit{Lisbon, April~2025}\\
\textit{NOVA University Lisbon}
