%!TEX root = ../template.tex

\chapter{Contribution Guidelines}
\label{chap:contribution}

\section{Overview}

The \gls{novathesis} project is a collaborative, open-source initiative maintained by the academic community at NOVA University Lisbon.  
Its continued evolution depends on feedback, bug reports, and contributions from users across institutions and disciplines.  
This chapter establishes the formal contribution workflow, coding standards, documentation requirements, and release procedures for maintainers and external contributors.

\section{Governance Model}

The project is maintained under the supervision of the NOVA University Lisbon repository administrators.  
Governance follows a meritocratic model typical of open academic projects:
\begin{itemize}
  \item Core maintainers have commit privileges and review merge requests;
  \item Regular contributors may submit patches or pull requests;
  \item Institutional representatives may propose new style presets or language modules.
\end{itemize}

Decisions regarding structural changes or new features are made by consensus among maintainers, with public discussion recorded in the project issue tracker.

\section{Repository Structure}

The Git repository follows a standardized hierarchy aligned with the template itself:

\begin{verbatim}
novathesis/
├── \gls{novathesis}Files/      # Core class and institutional presets
├── 0-Config/             # Default configuration templates
├── examples/             # Demonstration projects
├── docs/                 # Manual and auxiliary documentation
├── Makefile              # Build automation
├── README.md             # Overview and usage instructions
└── LICENSE.txt           # LPPL license declaration
\end{verbatim}

Contributors must ensure that new or modified files comply with this structure and do not introduce redundant or overlapping functionality.

\section{Code of Conduct}

All contributors are expected to adhere to the project’s Code of Conduct, based on the Contributor Covenant v2.1.  
In particular:
\begin{itemize}
  \item Be respectful and constructive in all discussions;
  \item Provide technically grounded feedback;
  \item Avoid personal, political, or institutional bias in reviews;
  \item Cite authoritative sources when proposing standard or typographic changes.
\end{itemize}

Violations of the Code of Conduct may result in suspension of contribution privileges.

\section{How to Contribute}

\subsection{Reporting Issues}

Before submitting a new issue, verify that it has not been previously reported.  
Use the issue tracker to file:
\begin{itemize}
  \item Bug reports (unexpected behavior, compilation errors);
  \item Feature requests (new configuration keys, institutional options);
  \item Documentation improvements or clarifications.
\end{itemize}

\paragraph{Bug Reports Should Include:}
\begin{itemize}
  \item \gls{novathesis} version and date (from the class header);
  \item Operating system and \LaTeX{} distribution;
  \item Compilation engine (\texttt{pdfLaTeX}, \texttt{XeLaTeX}, or \texttt{LuaLaTeX});
  \item A minimal working example reproducing the issue.
\end{itemize}

Example command to extract version information:
\begin{verbatim}
grep ProvidesClass \gls{novathesis}Files/novathesis.cls
\end{verbatim}

\subsection{Submitting Patches or Pull Requests}

All contributions must be submitted via GitHub pull requests.  
Each pull request must:
\begin{itemize}
  \item Target the \texttt{develop} branch (never \texttt{main});
  \item Contain atomic commits addressing a single logical change;
  \item Pass automated build and regression tests;
  \item Include updated documentation in the relevant configuration files.
\end{itemize}

\paragraph{Standard Procedure.}
\begin{enumerate}
  \item Fork the repository on GitHub;
  \item Create a feature branch:
    \begin{verbatim}
    git checkout -b feature/new-config-key
    \end{verbatim}
  \item Implement and test the modification;
  \item Commit with a descriptive message:
    \begin{verbatim}
    git commit -m "Add institution-specific cover page layout"
    \end{verbatim}
  \item Push the branch and open a pull request targeting \texttt{develop}.
\end{enumerate}

\paragraph{Commit Message Guidelines.}
\begin{itemize}
  \item Use the imperative mood (“Add”, “Fix”, “Update”) in subject lines;
  \item Keep the first line under 72 characters;
  \item Reference related issue numbers where applicable (e.g., “Fixes \#47”);
  \item Include a short rationale in the body if necessary.
\end{itemize}

\section{Development Standards}

\subsection{Class and Package Modifications}

When modifying or extending \texttt{novathesis.cls} or other internal files:
\begin{itemize}
  \item Maintain full backward compatibility with existing configuration keys;
  \item Do not hardcode institutional data or language-dependent strings;
  \item Use conditional macros and configuration hooks for all institution-specific logic;
  \item Document all new keys in \texttt{docs/Manual} or inline comments.
\end{itemize}

\paragraph{Naming Conventions.}
\begin{itemize}
  \item Class-level macros: \verb|\nt...| prefix (e.g., \verb|\ntsetup|, \verb|\ntauthor|);
  \item Internal variables: \verb|\NT@...| prefix (for maintainers only);
  \item File naming: lowercase, hyphen-separated (e.g., \texttt{nt-cover-ulisboa.sty}).
\end{itemize}

\subsection{Documentation Updates}

Every feature addition or configuration change must include corresponding updates to:
\begin{itemize}
  \item The \texttt{README.md} file;
  \item The User Manual (this document);
  \item The \texttt{CHANGELOG.md} file summarizing notable revisions.
\end{itemize}

All examples provided in the manual must compile without modification under the current release.

\subsection{Testing}

Each contribution must be tested under the following configurations:
\begin{itemize}
  \item TeX~Live, MacTeX, and MiKTeX distributions;
  \item \texttt{pdfLaTeX}, \texttt{XeLaTeX}, and \texttt{LuaLaTeX} engines;
  \item Linux, macOS, and Windows environments.
\end{itemize}

The repository includes a Makefile target for automated tests:
\begin{verbatim}
make test
\end{verbatim}

This command verifies successful compilation of all example documents.

\section{Institutional Preset Contributions}

Institutions wishing to integrate their formatting guidelines into \gls{novathesis} should follow the structured preset model.

\paragraph{Procedure.}
\begin{enumerate}
  \item Create a new folder under \texttt{\gls{novathesis}Files/Schools/};
  \item Copy and adapt an existing preset (e.g., \texttt{UNL}, \texttt{FCT});
  \item Define visual identity files (logo, colors, cover design);
  \item Update localized strings in \texttt{\gls{novathesis}Files/Strings/};
  \item Test the preset with a sample thesis project;
  \item Submit the preset as a pull request with accompanying documentation.
\end{enumerate}

Each preset must comply with the institutional graphic charter and \gls{novathesis} structure.

\section{Versioning and Release Workflow}

The project uses \textbf{semantic versioning (SemVer)} in the form \texttt{MAJOR.MINOR.PATCH}, e.g., \texttt{7.6.0}.

\paragraph{Policy.}
\begin{itemize}
  \item Increment the \texttt{MAJOR} version for incompatible API or class changes;
  \item Increment the \texttt{MINOR} version for new features that preserve compatibility;
  \item Increment the \texttt{PATCH} version for bug fixes and minor improvements.
\end{itemize}

\paragraph{Release Process.}
\begin{enumerate}
  \item Merge the \texttt{develop} branch into \texttt{main};
  \item Tag the release:
    \begin{verbatim}
    git tag -a v7.6.1 -m "Minor corrections and documentation update"
    \end{verbatim}
  \item Push tags:
    \begin{verbatim}
    git push --tags
    \end{verbatim}
  \item Generate release notes and update the manual version header.
\end{enumerate}

\section{License and Intellectual Property}

The \gls{novathesis} Template and all related files are distributed under the \textbf{LaTeX Project Public License (LPPL)} version~1.3c or later.  
All contributions are accepted under this license to ensure consistent legal status and free academic use.

\paragraph{Contributor Agreement.}
By submitting a contribution, you agree that:
\begin{itemize}
  \item Your work is original and does not infringe third-party rights;
  \item You license it under the LPPL for redistribution and modification;
  \item The maintainers may modify or redistribute your contribution under the same terms.
\end{itemize}

\paragraph{License Reference.}
\begin{quote}
LaTeX Project Public License (LPPL), version 1.3c or later.  
\url{https://www.latex-project.org/lppl.txt}
\end{quote}

\section{Citation and Acknowledgment of Contributions}

Contributors are credited in the release notes and documentation.  
Major institutional or code contributions should be cited formally in the project’s reference documentation as:

\begin{quote}
NOVA University Lisbon. \emph{\gls{novathesis} Template and User Manual}. Version~7.6.0, April~2025.  
Available at \url{https://github.com/novathesis/novathesis}.
\end{quote}

Contributors may also include their names in the internal \texttt{AUTHORS.md} file with institutional affiliation and contribution description.

\section{Communication Channels}

\begin{itemize}
  \item \textbf{GitHub Issues:} primary platform for bug reports and feature requests;
  \item \textbf{GitHub Discussions:} for usage questions and community support;
  \item \textbf{Mailing List:} maintained by the \gls{novathesis} coordination team for institutional correspondence;
  \item \textbf{Release Announcements:} published via GitHub Releases and the official university website.
\end{itemize}

\section{Deprecation and Backward Compatibility}

\gls{novathesis} maintains backward compatibility for at least two minor versions.  
Deprecated configuration keys are retained with warning messages before removal.

\paragraph{Policy.}
\begin{itemize}
  \item Deprecated keys trigger a log warning: “\texttt{WARNING: Deprecated option in novathesis}”;
  \item Equivalent replacement keys are documented in the changelog;
  \item Obsolete features are removed only in major version updates.
\end{itemize}

\section{Security and Integrity}

To maintain trust in official releases:
\begin{itemize}
  \item All release archives are signed using GPG keys of the maintainers;
  \item Checksums (SHA-256) are published alongside release notes;
  \item Users are encouraged to verify signatures before installation.
\end{itemize}

\section{Summary}

The \gls{novathesis} project thrives on community collaboration and academic transparency.  
These guidelines ensure that every contribution—whether a bug fix, feature enhancement, or institutional preset—is reviewed, tested, documented, and released consistently.  
By adhering to the governance, testing, and licensing policies described herein, contributors help maintain the template’s reliability, interoperability, and long-term sustainability across institutions and disciplines.
