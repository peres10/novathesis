%!TEX root = ../template.tex

\chapter{Bibliography and Citation Management}
\label{chap:bibliography}

\section{Overview}

The management of bibliographic references is a critical aspect of any academic thesis.  
\gls{novathesis} integrates the \texttt{biblatex} package with the \texttt{biber} backend, offering a modern, flexible, and language-aware citation system.  
This configuration provides full control over reference styles, multilingual support, and compliance with institutional formatting guidelines.

This chapter explains how to define, cite, and format bibliographic data within the \gls{novathesis} environment, including examples of both numeric and author–year citation schemes.

\section{Bibliography Architecture in \gls{novathesis}}

Bibliographic configuration in \gls{novathesis} is defined across three primary components:

\begin{enumerate}
  \item The global \texttt{biblatex} setup in \texttt{0-Config/3\_bibliography.tex};
  \item The bibliography database files (\texttt{.bib}) stored in \texttt{4-Bibliography/};
  \item The bibliography inclusion command in \texttt{3-BackMatter/appendices.tex} or \texttt{main.tex}.
\end{enumerate}

The system automatically manages language localization, label formatting, and cross-references between citations and bibliography entries.

\section{Default Configuration}

By default, \gls{novathesis} loads the following settings:

\begin{verbatim}
\usepackage[
  backend=biber,
  style=authoryear,
  citestyle=authoryear,
  sorting=nyt,
  maxbibnames=99,
  giveninits=true,
  isbn=false,
  url=true,
  doi=true,
  eprint=true
]{biblatex}

\addbibresource{4-Bibliography/references.bib}
\end{verbatim}

This configuration produces an author–year citation style compliant with most academic standards at NOVA University Lisbon.

\paragraph{Main Features.}
\begin{itemize}
  \item Full UTF-8 support for multilingual references;
  \item Automatic hyperlinking to citations and DOIs;
  \item Compact author lists with initials for first names;
  \item Sorting by name, year, and title;
  \item Automatic handling of et~al. abbreviations.
\end{itemize}

\section{Bibliography Database Structure}

Each entry in a \texttt{.bib} file follows the standard Bib\TeX{} syntax:

\begin{verbatim}
@article{smith2024energy,
  author    = {John Smith and Maria Silva},
  title     = {Renewable Energy Forecasting with Neural Networks},
  journal   = {Journal of Sustainable Energy},
  year      = {2024},
  volume    = {18},
  number    = {2},
  pages     = {55--72},
  doi       = {10.1234/jse.2024.0182}
}
\end{verbatim}

Files may contain any number of entries, grouped by type (\texttt{@article}, \texttt{@book}, \texttt{@inproceedings}, etc.).  
It is best practice to maintain a single, consolidated file such as \texttt{references.bib}.

\paragraph{Recommended Encoding.}
Always save \texttt{.bib} files in UTF-8 encoding to ensure compatibility with non-English characters.

\section{Adding and Managing Bibliography Files}

To include multiple bibliography databases, use:
\begin{verbatim}
\addbibresource{4-Bibliography/books.bib}
\addbibresource{4-Bibliography/articles.bib}
\end{verbatim}

Relative paths are recommended for portability.  
Avoid absolute paths or spaces in filenames.

\section{Citation Commands}

\gls{novathesis} supports all standard \texttt{biblatex} citation commands.

\paragraph{Author–Year Style.}
\begin{verbatim}
According to \textcite{smith2024energy}, forecasting accuracy improves...
\parencite{smith2024energy}
\end{verbatim}

Output:
\begin{quote}
According to Smith and Silva~(2024), forecasting accuracy improves…  
(Smith and Silva, 2024)
\end{quote}

\paragraph{Numeric Style.}
If the numeric style is preferred, change the \texttt{style} and \texttt{citestyle} options:
\begin{verbatim}
\usepackage[backend=biber,style=numeric,citestyle=numeric]{biblatex}
\end{verbatim}

Then use:
\begin{verbatim}
\cite{smith2024energy}
\end{verbatim}

\paragraph{Additional Commands.}
\begin{description}
  \item[\texttt{\textbackslash textcite\{\}}] — author name in text, year in parentheses;
  \item[\texttt{\textbackslash parencite\{\}}] — full citation in parentheses;
  \item[\texttt{\textbackslash citeauthor\{\}}] — author name only;
  \item[\texttt{\textbackslash citeyear\{\}}] — year only;
  \item[\texttt{\textbackslash fullcite\{\}}] — full bibliographic entry inline.
\end{description}

\section{Printing the Bibliography}

The bibliography is printed automatically at the end of the document.  
To include it manually, add the following command in the appropriate section:

\begin{verbatim}
\printbibliography[heading=bibintoc,title={References}]
\end{verbatim}

This ensures the bibliography appears in both the document and the table of contents.  
The title will be localized according to the document language.

\section{Language and Localization}

\gls{novathesis} automatically adapts citation labels and reference titles to the current document language.  
The \texttt{babel} or \texttt{polyglossia} package ensures correct translation of terms such as “editor”, “volume”, “accessed on”, etc.

\paragraph{Example.}
\begin{verbatim}
\selectlanguage{portuguese}
\printbibliography[title={Referências Bibliográficas}]
\end{verbatim}

\paragraph{Multilingual Documents.}
When compiling multilingual theses, each bibliography entry retains its original language, while field labels are translated dynamically.

\section{Institutional Citation Styles}

Different faculties within NOVA University may specify preferred citation styles.  
\gls{novathesis} includes preconfigured styles for:
\begin{itemize}
  \item \texttt{authoryear} — default (APA-like);
  \item \texttt{numeric-comp} — compressed numeric citations;
  \item \texttt{ieee} — IEEE-style numeric citations for engineering;
  \item \texttt{chicago-authordate} — humanities and social sciences.
\end{itemize}

To switch styles, modify:
\begin{verbatim}
\usepackage[backend=biber,style=ieee]{biblatex}
\end{verbatim}

\paragraph{Institutional Presets.}
Some presets (e.g., \texttt{FCT}, \texttt{NOVAIMS}) apply their own style automatically via the \texttt{school} key in \texttt{1\_novathesis.tex}.

\section{Managing DOIs, URLs, and Access Dates}

\texttt{biblatex} automatically formats DOIs and URLs as hyperlinks when \texttt{hyperref} is active.

\paragraph{Example.}
\begin{verbatim}
url    = {https://doi.org/10.1234/abcd.2024},
urldate= {2024-10-03},
\end{verbatim}

To hide URLs or DOIs in printed versions:
\begin{verbatim}
\AtEveryBibitem{\clearfield{url}\clearfield{doi}}
\end{verbatim}

\section{Bibliography in Appendices or Separate Sections}

If a thesis contains multiple bibliographies (e.g., one per chapter or appendix), use:
\begin{verbatim}
\printbibliography[heading=subbibliography,title={Chapter 2 References},segment=2]
\end{verbatim}

This requires enabling segmented bibliographies in \texttt{biblatex}:
\begin{verbatim}
\usepackage[refsegment=chapter]{biblatex}
\end{verbatim}

\section{Customizing the Bibliography Layout}

Formatting of the bibliography list can be customized through \texttt{biblatex} options and \gls{novathesis} configuration files.

\paragraph{Changing Font or Spacing.}
\begin{verbatim}
\setlength\bibitemsep{0.8\baselineskip}
\renewcommand*{\bibfont}{\small}
\end{verbatim}

\paragraph{Adding a Separator Line.}
\begin{verbatim}
\defbibheading{bibintoc}{%
  \section*{#1}\markboth{#1}{#1}\vspace{0.5cm}\hrule\vspace{1cm}}
\end{verbatim}

\section{Integration with Cross-Referencing}

All citation commands are fully compatible with the \texttt{hyperref} and \texttt{cleveref} systems.  
When clicking on a citation, readers are directed to the corresponding entry in the bibliography section.  
Back-references (from bibliography to text) can be enabled with:

\begin{verbatim}
\usepackage[backref=true]{biblatex}
\end{verbatim}

This feature is particularly useful in long documents or institutional reviews.

\section{Compiling the Bibliography}

\paragraph{Standard Workflow.}
\gls{novathesis} compiles the bibliography automatically when using the \texttt{Makefile}.  
To compile manually:

\begin{verbatim}
xelatex template.tex
biber template
xelatex template.tex
xelatex template.tex
\end{verbatim}

\paragraph{Overleaf.}
Overleaf automatically detects \texttt{biber} as the backend.  
However, in some cases, users must manually set the bibliography tool to “Biber” under the project’s Compiler settings.  
For large bibliographies, note that the free Overleaf tier may exceed processing limits.

\section{Validation and Troubleshooting}

\paragraph{Common Issues.}
\begin{itemize}
  \item \textbf{Undefined citations:} Re-run \texttt{biber} after the first \LaTeX{} compilation.
  \item \textbf{Character encoding errors:} Ensure \texttt{.bib} files are UTF-8 encoded.
  \item \textbf{Duplicate entries:} Each citation key must be unique across all databases.
  \item \textbf{Incorrect sorting:} Adjust the \texttt{sorting} option (e.g., \texttt{nyt}, \texttt{ynt}, or \texttt{none}).
\end{itemize}

\paragraph{Debugging.}
Use the following commands to inspect bibliography logs:
\begin{verbatim}
biber --debug template
cat template.blg
\end{verbatim}

\section{Best Practices}

\begin{enumerate}
  \item Maintain a single master \texttt{.bib} file under version control;
  \item Always include a DOI or URL when available;
  \item Use sentence case for titles unless required otherwise;
  \item Keep citation keys descriptive (e.g., \texttt{lastnameYYYYkeyword});
  \item Verify consistency in author name spelling across entries;
  \item Test the bibliography under both screen and print modes.
\end{enumerate}

\section{Summary}

\gls{novathesis} offers a modern and extensible bibliography management system based on \texttt{biblatex} and \texttt{biber}, ensuring consistency, multilingual adaptability, and compliance with institutional citation standards.  
By following the configuration and best practices outlined in this chapter, authors can produce clean, fully linked bibliographies that integrate seamlessly with the thesis structure and metadata system.
