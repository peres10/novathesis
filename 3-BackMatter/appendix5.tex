%!TEX root = ../template.tex

\chapter{Adding a New Font Theme}
\label{chap:newfont}

\section{Overview}

Typography plays a central role in the visual identity and readability of academic documents.  
NOVAthesis employs a modular system for font management, allowing users and institutions to define distinct \emph{font themes} that encapsulate all typographic parameters — including text, math, and monospaced fonts.  

This chapter explains how to create, configure, and test a new font theme using the NOVAthesis modular system, without modifying the core class.  
It also describes recommended font combinations and institutional customization guidelines.

\section{Font Theme Architecture}

All font-related configurations are located in the folder:

\begin{quote}
\texttt{NOVAthesisFiles/Fonts/}
\end{quote}

Each font theme is stored in an independent file, typically named:

\begin{verbatim}
NOVAthesisFiles/Fonts/nt-font-<theme>.sty
\end{verbatim}

Examples included with the distribution:
\begin{verbatim}
nt-font-default.sty
nt-font-modern.sty
nt-font-times.sty
nt-font-sans.sty
nt-font-palatino.sty
\end{verbatim}

Font themes are selected using the configuration key:
\begin{verbatim}
\ntsetup{font=<theme>}
\end{verbatim}

\paragraph{Example.}
\begin{verbatim}
\ntsetup{
  font=modern,
  mode=final,
  media=screen
}
\end{verbatim}

This command loads \texttt{NOVAthesisFiles/Fonts/nt-font-modern.sty} automatically.

\section{Structure of a Font Theme File}

Each font theme file defines three categories of fonts:

\begin{enumerate}
  \item Serif (main text);
  \item Sans-serif (headings, captions, or alternate);
  \item Monospace (code or listings).
\end{enumerate}

In addition, the theme specifies math font compatibility and general typographic spacing parameters.

\paragraph{Example Skeleton.}
\begin{verbatim}
% nt-font-minion.sty — NOVAthesis font theme (Minion + Myriad)
% Encoding: UTF-8
\ProvidesPackage{nt-font-minion}[2025/04/18 NOVAthesis font theme: Minion/Myriad]

% Load required packages
\RequirePackage{fontspec}

% Serif font (main text)
\setmainfont{Minion Pro}[
  Ligatures=TeX,
  Numbers=OldStyle,
  Scale=1.0
]

% Sans-serif font (headings)
\setsansfont{Myriad Pro}[
  Ligatures=TeX,
  Scale=MatchLowercase
]

% Monospace font (code, listings)
\setmonofont{Inconsolata}[
  Scale=MatchLowercase
]

% Math font (if available)
\setmathfont{Minion Math}

% Adjust paragraph spacing and leading
\linespread{1.05}
\end{verbatim}

All commands are executed within the scope of the template’s internal font setup module.  
If a font is unavailable on the system, XeLaTeX or LuaLaTeX will issue a warning and fall back to the default.

\section{Font Theme Naming Conventions}

\begin{itemize}
  \item The filename must begin with \texttt{nt-font-};
  \item The extension must be \texttt{.sty};
  \item The internal \texttt{\textbackslash ProvidesPackage} declaration must match the file name;
  \item The theme name used in \texttt{\textbackslash ntsetup\{font=\ldots\}} must correspond to the suffix of the file name.
\end{itemize}

\paragraph{Example.}
\begin{verbatim}
File: nt-font-garamond.sty
Called as: \ntsetup{font=garamond}
\end{verbatim}

\section{Recommended Fonts and Compatibility}

NOVAthesis officially supports Unicode-aware fonts available in TeX~Live and modern operating systems.

\paragraph{Recommended Typefaces.}
\begin{itemize}
  \item \textbf{Times New Roman / TeX Gyre Termes} — traditional, widely accepted;
  \item \textbf{Palatino / TeX Gyre Pagella} — balanced for long reading;
  \item \textbf{Minion + Myriad} — professional dual-family pair;
  \item \textbf{Libertinus} — free, open-source family with full math coverage;
  \item \textbf{STIX Two} — excellent for scientific publications;
  \item \textbf{Latin Modern} — default fallback, compatible with all TeX distributions.
\end{itemize}

\paragraph{Free and Open Source Fonts.}
For maximum portability, it is recommended that font themes rely on open-source fonts included in TeX~Live.

\section{Math Font Configuration}

Mathematical fonts must be compatible with the main text font to ensure visual harmony.

\paragraph{Example (Libertinus).}
\begin{verbatim}
\setmainfont{Libertinus Serif}
\setsansfont{Libertinus Sans}
\setmonofont{Libertinus Mono}
\setmathfont{Libertinus Math}
\end{verbatim}

\paragraph{Example (TeX Gyre).}
\begin{verbatim}
\setmainfont{TeX Gyre Pagella}
\setmathfont{TeX Gyre Pagella Math}
\end{verbatim}

If no math font is declared, NOVAthesis defaults to Latin Modern Math.

\section{Conditional Font Loading}

To provide fallback behavior, conditionals can be included to verify whether the desired font is installed.

\paragraph{Example.}
\begin{verbatim}
\IfFontExistsTF{Minion Pro}{
  \setmainfont{Minion Pro}
}{
  \setmainfont{TeX Gyre Pagella}
}
\end{verbatim}

This ensures robust compilation even on systems lacking proprietary fonts.

\section{Fine-Tuning Typography}

Each font theme may redefine typographic parameters such as:
\begin{itemize}
  \item \texttt{\textbackslash linespread} — line spacing multiplier;
  \item \texttt{\textbackslash baselinestretch} — alternative for line height control;
  \item \texttt{\textbackslash parskip} and \texttt{\textbackslash parindent};
  \item Title and section font weights using \texttt{\textbackslash renewcommand};
  \item Small-caps or letter-spacing adjustments for headings.
\end{itemize}

\paragraph{Example (adjust headings).}
\begin{verbatim}
\renewcommand{\chapternamefont}{\sffamily\Large\bfseries}
\renewcommand{\chapternumberfont}{\sffamily\huge\bfseries}
\renewcommand{\chaptertitlefont}{\sffamily\Huge\bfseries}
\end{verbatim}

\section{Testing and Validation}

Before finalizing a new font theme, test it under all standard configurations.

\paragraph{Test Checklist.}
\begin{itemize}
  \item [ ] Compile with XeLaTeX and LuaLaTeX (both must succeed);
  \item [ ] Verify that all fonts load correctly (no “font not found” warnings);
  \item [ ] Inspect the title page, table of contents, and bibliography;
  \item [ ] Confirm proper math symbol alignment and equation spacing;
  \item [ ] Ensure small-caps and bold weights render consistently;
  \item [ ] Check PDF font embedding using:
    \begin{verbatim}
    pdffonts template.pdf
    \end{verbatim}
\end{itemize}

\paragraph{Compilation Example.}
\begin{verbatim}
make MODE=final MEDIA=paper font=garamond xe
\end{verbatim}

\section{Packaging and Contribution Guidelines}

To distribute a new font theme with the template or submit it for inclusion in the official repository, follow these steps:

\begin{enumerate}
  \item Place the file under \texttt{NOVAthesisFiles/Fonts/};
  \item Ensure it includes a proper header with author, version, and license;
  \item Add a comment specifying font licensing and installation requirements;
  \item Test on both macOS (MacTeX) and Linux (TeX~Live);
  \item Submit a pull request including:
    \begin{itemize}
      \item The new theme file (\texttt{nt-font-<name>.sty});
      \item A compiled example thesis using the theme;
      \item A short entry in the documentation (this manual or \texttt{README.md}).
    \end{itemize}
\end{enumerate}

\section{Licensing and Font Distribution}

Due to licensing restrictions, NOVAthesis does not distribute proprietary fonts.  
Users are responsible for ensuring that any commercial typeface (e.g., Minion Pro, Myriad Pro) is properly licensed on their system.  

Open-source themes (e.g., Libertinus, TeX~Gyre, STIX) may be redistributed under their respective licenses, provided that the NOVAthesis class remains under LPPL.

\section{Troubleshooting}

\paragraph{Common Issues.}
\begin{itemize}
  \item \textbf{Font not found:} Ensure the font is installed system-wide and accessible to XeLaTeX or LuaLaTeX;
  \item \textbf{Incorrect bold or italic variants:} Specify explicit \texttt{BoldFont=} and \texttt{ItalicFont=} options;
  \item \textbf{Math symbols mismatched:} Use a compatible math font or revert to Latin Modern Math;
  \item \textbf{Inconsistent spacing:} Adjust \texttt{\textbackslash linespread} and \texttt{\textbackslash parskip} values.
\end{itemize}

\paragraph{Debugging Fonts.}
Use:
\begin{verbatim}
luaotfload-tool --find "Font Name"
\end{verbatim}
to confirm font availability, or:
\begin{verbatim}
fc-list | grep "Font Name"
\end{verbatim}
on Linux/macOS systems.

\section{Summary}

Font themes in NOVAthesis are modular, portable, and easy to customize.  
By defining a new \texttt{nt-font-<name>.sty} file within the \texttt{NOVAthesisFiles/Fonts/} directory, users can achieve institution-specific typography while maintaining full compatibility with the template’s automation and build system.  

Following the conventions and testing procedures described in this chapter ensures that new typographic configurations remain consistent, reproducible, and compliant with academic publishing standards.
