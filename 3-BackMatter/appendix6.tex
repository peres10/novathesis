%!TEX root = ../template.tex

\chapter{Adding a New Chapter Theme}
\label{chap:newchaptertheme}

\section{Overview}

NOVAthesis uses a modular system to define the visual appearance of chapter and section titles.  
These designs, known as \emph{chapter themes}, determine how chapter numbers, titles, subtitles, and spacing are formatted throughout the document.

A chapter theme defines not only typography, but also geometry, alignment, color, and decorative elements (such as rules, boxes, or background panels).  
This modular approach allows institutions to implement specific visual identities without altering the core class or template logic.

This chapter describes how to create a new chapter theme, integrate it into the existing system, and test it for visual consistency and compliance with academic standards.

\section{Theme Architecture and File Structure}

All chapter themes are stored in the directory:

\begin{quote}
\texttt{NOVAthesisFiles/ChapterStyles/}
\end{quote}

Each file defines a single theme using the naming convention:

\begin{verbatim}
nt-chapter-<theme>.sty
\end{verbatim}

Examples distributed with the template include:

\begin{verbatim}
nt-chapter-default.sty
nt-chapter-modern.sty
nt-chapter-simple.sty
nt-chapter-fct.sty
\end{verbatim}

Themes are selected via the configuration key:

\begin{verbatim}
\ntsetup{chapterstyle=<theme>}
\end{verbatim}

\paragraph{Example.}
\begin{verbatim}
\ntsetup{
  chapterstyle=modern,
  font=palatino,
  mode=final
}
\end{verbatim}

This command automatically loads \texttt{NOVAthesisFiles/ChapterStyles/nt-chapter-modern.sty}.

\section{Structure of a Chapter Theme File}

Each chapter theme file defines:
\begin{enumerate}
  \item The style for chapter titles;
  \item Optional redefinitions for section, subsection, and part headings;
  \item Visual spacing and alignment rules;
  \item Optional background or decorative elements.
\end{enumerate}

\paragraph{Example Skeleton.}
\begin{verbatim}
% nt-chapter-minimal.sty — NOVAthesis chapter theme (Minimal)
% Encoding: UTF-8
\ProvidesPackage{nt-chapter-minimal}[2025/04/20 Minimal chapter style for NOVAthesis]

% 1. Load dependencies
\RequirePackage{titlesec}
\RequirePackage{xcolor}

% 2. Define color scheme
\definecolor{ChapterColor}{RGB}{0, 55, 133}

% 3. Chapter title format
\titleformat{\chapter}[display]
  {\normalfont\huge\bfseries\color{ChapterColor}}
  {\filcenter\MakeUppercase{\chaptername}~\thechapter}
  {0.5em}
  {\filcenter}
  [\vspace{1em}\titlerule]

% 4. Chapter spacing
\titlespacing*{\chapter}{0pt}{-2em}{3em}

% 5. Section and subsection styles (optional)
\titleformat{\section}
  {\normalfont\Large\bfseries\color{ChapterColor}}
  {\thesection}{1em}{}

\titlespacing*{\section}{0pt}{2em}{1em}
\end{verbatim}

This example defines a clean, centered layout with a colored chapter number and a horizontal rule below the title.

\section{Naming Conventions}

\begin{itemize}
  \item File names must follow the pattern \texttt{nt-chapter-<name>.sty};
  \item The internal \texttt{\textbackslash ProvidesPackage} declaration must match the file name;
  \item The value of \texttt{<name>} is used as the key for \texttt{\textbackslash ntsetup\{chapterstyle=<name>\}};
  \item Each theme must be self-contained and must not redefine unrelated elements.
\end{itemize}

\paragraph{Example.}
\begin{verbatim}
File: nt-chapter-academic.sty
Call: \ntsetup{chapterstyle=academic}
\end{verbatim}

\section{Customization Parameters}

NOVAthesis provides configuration hooks for chapter formatting through \texttt{titlesec}.  
Themes can adjust the following parameters:

\begin{description}
  \item[\texttt{display}] — show the chapter number and title on separate lines;
  \item[\texttt{hang}] — indent title text under the number label;
  \item[\texttt{runin}] — display number and title on the same line;
  \item[\texttt{block}] — align number and title in a framed block.
\end{description}

\paragraph{Example (Left-Aligned Style).}
\begin{verbatim}
\titleformat{\chapter}[hang]
  {\normalfont\Huge\bfseries}
  {\thechapter.}{1em}{}
\titlespacing*{\chapter}{0pt}{-1em}{1em}
\end{verbatim}

\section{Integration with Institutional Presets}

Institutions may define their own official chapter style files under:

\begin{verbatim}
NOVAthesisFiles/Schools/<Institution>/ChapterStyles/
\end{verbatim}

Each institutional preset can reference its preferred theme using:

\begin{verbatim}
\ntsetup{chapterstyle=fct}
\end{verbatim}

During compilation, NOVAthesis checks for a matching theme in the institutional folder before loading the default from \texttt{ChapterStyles/}.

\section{Color and Font Coordination}

To maintain visual coherence, each chapter theme should use the same color palette and font configuration as the selected \texttt{font} and \texttt{color} themes.

\paragraph{Example (Coordinated Style).}
\begin{verbatim}
\definecolor{InstitutionColor}{RGB}{15, 85, 150}

\titleformat{\chapter}[display]
  {\normalfont\huge\bfseries\color{InstitutionColor}\sffamily}
  {\MakeUppercase{\chaptername}~\thechapter}
  {0.5em}
  {\filcenter}
  [\vspace{0.5em}\titlerule]
\end{verbatim}

When a theme specifies colors or fonts, it should respect the currently active font theme and color scheme unless explicitly overridden.

\section{Optional Subtitle Support}

Some institutions include optional chapter subtitles.  
These can be defined by extending the formatting command:

\begin{verbatim}
\titleformat{name=\chapter,numberless}[display]
  {\normalfont\Huge\bfseries}
  {}
  {0pt}
  {\filcenter}
  [\vspace{1em}\Large\itshape\@subtitle]
\end{verbatim}

The subtitle variable (\texttt{\@subtitle}) can be set via:

\begin{verbatim}
\renewcommand{\@subtitle}{Optional subtitle text}
\end{verbatim}

\section{Advanced Design Techniques}

More elaborate themes may incorporate graphical or decorative elements, such as background shapes or colored boxes.

\paragraph{Example (Framed Chapter Title).}
\begin{verbatim}
\RequirePackage{tikz}

\titleformat{\chapter}[display]
  {\normalfont\Huge\bfseries}
  {}
  {0pt}{
    \begin{tikzpicture}[remember picture,overlay]
      \node[fill=gray!20,rounded corners=5pt,
        inner sep=10pt,text width=\textwidth-2cm]{
        \centering\color{black}\chaptername~\thechapter\\[5pt]
        \Large\bfseries\MakeUppercase{#1}
      };
    \end{tikzpicture}
  }
\end{verbatim}

Themes using TikZ or other graphical elements must load their packages locally to avoid conflicts with other modules.

\section{Testing and Validation}

Before finalizing a new chapter theme, perform a visual inspection of:

\begin{itemize}
  \item [ ] Chapter title spacing and alignment;
  \item [ ] Consistent number formatting and case usage;
  \item [ ] Page break behavior before and after chapter openings;
  \item [ ] Compatibility with section headings, TOC entries, and bookmarks;
  \item [ ] Consistency in print and screen modes.
\end{itemize}

\paragraph{Recommended Test Build.}
\begin{verbatim}
make MODE=provisional chapterstyle=academic xe
\end{verbatim}

\paragraph{Preview Mode.}
Use:
\begin{verbatim}
make preview
\end{verbatim}
to compile quickly without glossary or bibliography.

\section{Integration with the NOVAthesis Configuration System}

Each theme is automatically registered when placed in the correct folder.  
The class loads the file according to the following order of precedence:

\begin{enumerate}
  \item Custom file in the project’s local \texttt{0-Config/} folder;
  \item Institutional theme in \texttt{NOVAthesisFiles/Schools/<School>/ChapterStyles/};
  \item Generic theme in \texttt{NOVAthesisFiles/ChapterStyles/};
  \item Fallback to \texttt{nt-chapter-default.sty}.
\end{enumerate}

This hierarchy allows safe institutional overrides while maintaining backward compatibility.

\section{Packaging and Contribution Guidelines}

To contribute a new chapter theme to the official repository:

\begin{enumerate}
  \item Place the file in \texttt{NOVAthesisFiles/ChapterStyles/};
  \item Include a descriptive header with version, author, and date;
  \item Follow naming conventions and coding standards shown above;
  \item Include a sample PDF demonstrating the visual layout;
  \item Submit a pull request referencing the NOVAthesis version.
\end{enumerate}

All contributions must comply with the LPPL license and include only original design work or redistributable graphical assets.

\section{Troubleshooting}

\paragraph{Common Issues.}
\begin{itemize}
  \item \textbf{Chapter numbers missing:} Ensure the \texttt{titlesec} style includes \texttt{\textbackslash thechapter};
  \item \textbf{Inconsistent spacing:} Adjust vertical skips using \texttt{\textbackslash titlespacing*};
  \item \textbf{Color not applied:} Check that \texttt{xcolor} is loaded and the color name matches the theme palette;
  \item \textbf{Compilation errors:} Ensure the file uses UTF-8 encoding and does not redefine class macros directly.
\end{itemize}

\section{Summary}

NOVAthesis chapter themes provide a flexible mechanism for customizing the visual presentation of chapter titles while maintaining full structural compatibility with the class.  
By creating a new \texttt{nt-chapter-<name>.sty} file, users can implement institutional branding or stylistic preferences without modifying the template core.  
Adhering to the structure, naming conventions, and testing procedures described here ensures consistent, professional-quality output across all thesis projects.
