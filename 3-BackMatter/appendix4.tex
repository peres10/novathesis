%!TEX root = ../template.tex

\chapter{Adding a New Language}
\label{chap:newlanguage}

\section{Overview}

NOVAthesis is designed to support multilingual theses through an extensible localization framework.  
By default, it includes English and Portuguese translations for all interface elements (captions, lists, front matter labels, and metadata).  
However, additional languages can be added easily without altering the core class files.

This chapter describes how to define a new language, configure translations, adjust typographic conventions, and ensure full compatibility with \texttt{babel} or \texttt{polyglossia}.  
It also provides best practices for testing and maintaining new language modules.

\section{Localization Framework in NOVAthesis}

Localization in NOVAthesis is handled through modular string definition files located in:

\begin{quote}
\texttt{NOVAthesisFiles/Strings/}
\end{quote}

Each file defines the translation of all labels, captions, and terms required for the template.  
For example:

\begin{verbatim}
NOVAthesisFiles/Strings/english.ldf
NOVAthesisFiles/Strings/portuguese.ldf
\end{verbatim}

To add a new language, create a new file (e.g., \texttt{french.ldf}) and define the localized strings following the same structure.

\section{Creating a New Language File}

\paragraph{Step 1 — Create the File.}
Copy an existing language file and rename it according to the target language code:

\begin{verbatim}
cp NOVAthesisFiles/Strings/english.ldf NOVAthesisFiles/Strings/french.ldf
\end{verbatim}

\paragraph{Step 2 — Define Language Metadata.}
At the top of the file, declare the language properties:

\begin{verbatim}
% french.ldf — NOVAthesis language definition
% Encoding: UTF-8
\ProvidesFile{french.ldf}[2025/04/15 French translations for NOVAthesis]
\def\NT@langname{french}
\def\NT@langlabel{Français}
\end{verbatim}

\paragraph{Step 3 — Translate the Strings.}
Each string is defined with the command:
\begin{verbatim}
\ntstringdef{<key>}{<translation>}
\end{verbatim}

\paragraph{Example.}
\begin{verbatim}
\ntstringdef{abstractname}{Résumé}
\ntstringdef{acknowledgmentsname}{Remerciements}
\ntstringdef{contentsname}{Table des matières}
\ntstringdef{listfigurename}{Liste des figures}
\ntstringdef{listtablename}{Liste des tableaux}
\ntstringdef{bibname}{Références bibliographiques}
\ntstringdef{appendixname}{Annexe}
\ntstringdef{supervisorname}{Directeur de thèse}
\ntstringdef{degreeprefix}{Thèse de Doctorat}
\end{verbatim}

Ensure that all relevant keys are included.  
The full list of supported keys is documented in \texttt{english.ldf} and \texttt{portuguese.ldf}.

\paragraph{Step 4 — Save Using UTF-8 Encoding.}
All `.ldf` files must be saved in UTF-8 encoding to preserve special characters and diacritics.

\section{Registering the New Language}

After creating the file, the new language must be registered in the global configuration.

\paragraph{Step 1 — Declare the Language in the Main File.}
In \texttt{1\_novathesis.tex}, modify:

\begin{verbatim}
\ntsetup{
  mainlanguage=french,
  otherlanguages={english,portuguese}
}
\end{verbatim}

\paragraph{Step 2 — Ensure Package Compatibility.}
For XeLaTeX or LuaLaTeX, NOVAthesis uses \texttt{polyglossia}.  
For pdfLaTeX, it uses \texttt{babel}.  
Both require that the new language be supported by the system TeX distribution.

\paragraph{For \texttt{polyglossia}:}
\begin{verbatim}
\setdefaultlanguage{french}
\setotherlanguages{english,portuguese}
\end{verbatim}

\paragraph{For \texttt{babel}:}
\begin{verbatim}
\usepackage[french,english,portuguese]{babel}
\selectlanguage{french}
\end{verbatim}

If the target language is not included in TeX~Live, install its corresponding package (e.g., \texttt{texlive-lang-french}).

\section{Testing the New Language}

Compile the document with the new language set as the main language:

\begin{verbatim}
make MODE=working MEDIA=screen xe
\end{verbatim}

Verify the following:
\begin{itemize}
  \item The title page and front matter labels display correctly;
  \item Section titles (e.g., “Chapter”, “Figure”, “Table”) are localized;
  \item Lists of figures and tables use the correct captions;
  \item Glossary and bibliography headings appear in the target language;
  \item PDF metadata (Title, Author, Keywords) include the translated values.
\end{itemize}

\section{Multilingual Compilation}

NOVAthesis allows simultaneous inclusion of multiple languages in the same document.  
To use multiple languages dynamically:

\begin{verbatim}
\selectlanguage{english}
\chapter{Introduction}

\selectlanguage{french}
\chapter{Méthodologie}

\selectlanguage{portuguese}
\chapter{Resultados}
\end{verbatim}

When switching languages, NOVAthesis automatically updates labels, figure captions, and cross-reference text through its internal localization mechanism.

\section{Translating Metadata Fields}

Metadata such as titles, abstracts, and keywords must also be localized using language-specific keys.

\paragraph{Example.}
\begin{verbatim}
\nttitle(main,en){Machine Learning for Climate Prediction}
\nttitle(main,fr){Apprentissage automatique pour la prévision climatique}
\ntabstract(fr){
Cette thèse présente des modèles d'apprentissage automatique pour la prévision climatique.
}
\ntkeywords(fr){apprentissage, prévision, climat, intelligence artificielle}
\end{verbatim}

NOVAthesis automatically selects the correct variant based on the active language.

\section{Language-Specific Typography and Formatting}

Different languages require specific typographic conventions.  
When adding a new language, review the following parameters:

\begin{itemize}
  \item \textbf{Quotation marks:} adjust via \texttt{\textbackslash frenchspacing} or language-specific macros;
  \item \textbf{Date format:} redefine with \texttt{\textbackslash today};
  \item \textbf{Hyphenation patterns:} ensure the language dictionary is installed;
  \item \textbf{Decimal and thousands separators:} controlled by \texttt{siunitx};
  \item \textbf{Font choices:} select a font family that supports all required glyphs.
\end{itemize}

\paragraph{Example (French typography).}
\begin{verbatim}
\frenchspacing
\renewcommand{\today}{\number\day~\ifcase\month\or
janvier\or février\or mars\or avril\or mai\or juin\or
juillet\or août\or septembre\or octobre\or novembre\or décembre\fi~\number\year}
\end{verbatim}

\section{Right-to-Left and Non-Latin Languages}

For right-to-left (RTL) scripts such as Arabic, Hebrew, or Persian, compilation with \texttt{XeLaTeX} or \texttt{LuaLaTeX} is mandatory.  
NOVAthesis inherits directionality handling from \texttt{polyglossia}.

\paragraph{Example.}
\begin{verbatim}
\setmainlanguage{arabic}
\setotherlanguages{english}
\newfontfamily\arabicfont[Script=Arabic]{Amiri}
\end{verbatim}

RTL languages may also require localized class adaptations for page numbering, table alignment, and chapter heading direction.

\section{Integration with Glossaries and Bibliography}

When using glossaries and bibliographies in multiple languages, ensure that translations are provided for all glossary entries and that the bibliography style supports multilingual fields.

\paragraph{Glossary Example.}
\begin{verbatim}
\newglossaryentry{energy}{
  name={énergie},
  description={Capacité d'un système à effectuer un travail}
}
\end{verbatim}

\paragraph{Bibliography Example.}
\begin{verbatim}
@book{dupont2024ai,
  author  = {Jean Dupont},
  title   = {Apprentissage profond et systèmes intelligents},
  year    = {2024},
  langid  = {french},
  publisher = {Presses Universitaires de Paris}
}
\end{verbatim}

\texttt{biblatex} uses the \texttt{langid} field to format entries according to the target language.

\section{Testing and Validation Checklist}

Before releasing a new language file, verify the following:

\begin{itemize}
  \item [ ] All string keys defined in \texttt{english.ldf} are translated;
  \item [ ] File encoding is UTF-8;
  \item [ ] The language compiles under both XeLaTeX and pdfLaTeX;
  \item [ ] Accented characters and special punctuation render correctly;
  \item [ ] Captions, headings, and metadata appear in the target language;
  \item [ ] The glossary and bibliography correctly localize labels;
  \item [ ] Documentation includes acknowledgment of the translator.
\end{itemize}

\section{Submitting a New Language to the Project}

Once validated, new language files may be contributed to the official repository.

\paragraph{Procedure.}
\begin{enumerate}
  \item Fork the NOVAthesis repository;
  \item Add the new file under \texttt{NOVAthesisFiles/Strings/};
  \item Test with \texttt{make test};
  \item Update the documentation (in this manual);
  \item Submit a pull request with:
    \begin{itemize}
      \item The new \texttt{.ldf} file;
      \item Example configuration snippet;
      \item Translator’s name and institution.
    \end{itemize}
\end{enumerate}

\paragraph{License.}
Language contributions are accepted under the same LPPL license as the rest of the project.

\section{Summary}

Adding a new language to NOVAthesis is a straightforward and modular process.  
By creating a new \texttt{.ldf} file, registering it in the configuration, and testing it under multiple compilation engines, contributors can expand the template’s multilingual capabilities without altering its core logic.  
This approach promotes inclusivity, international collaboration, and full linguistic accessibility in academic publishing.
