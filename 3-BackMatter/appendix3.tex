%!TEX root = ../template.tex

\chapter{Example Makefile Targets}
\label{chap:makefile}

\section{Overview}

NOVAthesis provides a fully automated build system implemented through a \texttt{Makefile} located at the root of each project.  
This system abstracts all \LaTeX{} compilation steps, including bibliography generation, glossary creation, and PDF packaging, ensuring reproducibility and consistency across different environments.

The \texttt{Makefile} is compatible with \texttt{GNU~Make} on Linux, macOS, and Windows (via WSL, Git~Bash, or MiKTeX shell).  
It also supports automated builds on cloud platforms and continuous integration environments.

This chapter explains the standard targets available in the NOVAthesis \texttt{Makefile}, their dependencies, and typical use cases.

\section{Basic Compilation Targets}

The simplest way to compile a thesis is by invoking one of the pre-defined targets corresponding to the desired \LaTeX{} engine.

\paragraph{Examples.}
\begin{verbatim}
make pdf        # Compile using pdfLaTeX
make xe         # Compile using XeLaTeX
make lua        # Compile using LuaLaTeX
\end{verbatim}

Each command performs a full build cycle:
\begin{itemize}
  \item Invokes \texttt{latexmk} with appropriate options;
  \item Runs \texttt{biber} for bibliography processing;
  \item Executes \texttt{makeglossaries} if glossaries or acronyms are defined;
  \item Cleans and synchronizes all auxiliary files.
\end{itemize}

\paragraph{Default Engine.}
If no target is specified, NOVAthesis defaults to XeLaTeX:
\begin{verbatim}
make
\end{verbatim}

This ensures consistent font rendering and Unicode support across systems.

\section{Working, Provisional, and Final Modes}

The build mode controls visual features such as watermarking, TODO notes, and link colors.  
Modes are set using the \texttt{MODE} variable:

\begin{verbatim}
make MODE=working xe
make MODE=provisional xe
make MODE=final pdf
\end{verbatim}

\begin{description}
  \item[\texttt{working}] — enables draft markers, colored warnings, and visible notes;
  \item[\texttt{provisional}] — disables watermarks but retains colored hyperlinks;
  \item[\texttt{final}] — produces the definitive, submission-ready version.
\end{description}

\paragraph{Media Type.}
Select the intended output medium (screen or paper):

\begin{verbatim}
make MODE=final MEDIA=paper xe
make MODE=final MEDIA=screen xe
\end{verbatim}

\section{Bibliography and Glossary Management}

Bibliography and glossary generation are automated through dependent targets.

\paragraph{Bibliography Only.}
\begin{verbatim}
make bib
\end{verbatim}
Executes:
\begin{verbatim}
biber template
\end{verbatim}

\paragraph{Glossary Only.}
\begin{verbatim}
make gloss
\end{verbatim}
Executes:
\begin{verbatim}
makeglossaries template
\end{verbatim}

\paragraph{Combined Build.}
\begin{verbatim}
make all
\end{verbatim}
Runs a full compilation sequence:
\begin{verbatim}
latexmk -xelatex template
biber template
makeglossaries template
latexmk -xelatex template
\end{verbatim}

\section{Cleaning Targets}

To maintain a clean working directory, the Makefile defines several cleaning targets.

\paragraph{Remove Temporary Files.}
\begin{verbatim}
make clean
\end{verbatim}
Deletes auxiliary files (\texttt{.aux}, \texttt{.log}, \texttt{.out}, etc.) but preserves the final PDF.

\paragraph{Full Cleanup.}
\begin{verbatim}
make clean-all
\end{verbatim}
Removes all auxiliary and generated files, including:
\begin{itemize}
  \item Bibliography cache files (\texttt{.bbl}, \texttt{.bcf});
  \item Glossary and acronym outputs (\texttt{.gls}, \texttt{.glo});
  \item Auxiliary directories created by \texttt{latexmk}.
\end{itemize}

\paragraph{Target-Specific Cleanup.}
\begin{verbatim}
make clean-bib
make clean-gloss
\end{verbatim}

\section{Packaging and Distribution}

NOVAthesis includes automated packaging for archiving and submission.

\paragraph{Create a Submission Archive.}
\begin{verbatim}
make package
\end{verbatim}
Generates a ZIP archive containing:
\begin{itemize}
  \item The compiled PDF (\texttt{template.pdf});
  \item All configuration files and source directories;
  \item Figures and bibliography resources;
  \item A \texttt{README.txt} with version and metadata.
\end{itemize}

\paragraph{Exclude Source Files (PDF only).}
\begin{verbatim}
make package SRC=false
\end{verbatim}

\paragraph{Custom Output Directory.}
\begin{verbatim}
make OUTDIR=build xe
\end{verbatim}

\section{Version and Metadata Verification}

The Makefile includes a dedicated validation command that verifies all required metadata fields before compilation.

\paragraph{Check Metadata Completeness.}
\begin{verbatim}
make check-metadata
\end{verbatim}
This target parses \texttt{1\_novathesis.tex} and reports any missing definitions such as:
\begin{itemize}
  \item Author or title;
  \item Supervisor or degree name;
  \item Date or institution.
\end{itemize}

\paragraph{Check for Unused References.}
\begin{verbatim}
make check-refs
\end{verbatim}
Scans for undefined or unused bibliography keys.

\section{Testing and Continuous Integration Targets}

For maintainers and automated systems, the Makefile provides non-interactive test modes.

\paragraph{Automated Validation Build.}
\begin{verbatim}
make test
\end{verbatim}
Compiles the template using XeLaTeX, reports errors, and validates all example projects.

\paragraph{Silent CI Mode.}
\begin{verbatim}
make ci
\end{verbatim}
Equivalent to:
\begin{verbatim}
latexmk -xelatex -interaction=nonstopmode -halt-on-error
\end{verbatim}
Useful for integration with GitHub Actions or GitLab CI pipelines.

\section{Diagnostic and Utility Targets}

\paragraph{View Log File.}
\begin{verbatim}
make log
\end{verbatim}
Displays the most recent \texttt{.log} file in the terminal.

\paragraph{Word Count Estimate.}
\begin{verbatim}
make count
\end{verbatim}
Uses \texttt{texcount} to generate a detailed word count by chapter.

\paragraph{Check PDF Metadata.}
\begin{verbatim}
make pdfinfo
\end{verbatim}
Runs:
\begin{verbatim}
pdfinfo template.pdf
\end{verbatim}

\paragraph{Preview Build (draft mode).}
\begin{verbatim}
make preview
\end{verbatim}
Runs a fast build without bibliography or glossary updates for rapid iteration.

\section{Customization of Makefile Variables}

The Makefile accepts environment variables to control its behavior.

\begin{description}
  \item[\texttt{MODE}] — build mode: \texttt{working}, \texttt{provisional}, or \texttt{final};
  \item[\texttt{MEDIA}] — output type: \texttt{screen} or \texttt{paper};
  \item[\texttt{ENGINE}] — compiler: \texttt{pdf}, \texttt{xe}, or \texttt{lua};
  \item[\texttt{OUTDIR}] — directory for compiled outputs (default: project root);
  \item[\texttt{SRC}] — include source files in package (boolean);
  \item[\texttt{VERBOSE}] — show or suppress log output.
\end{description}

\paragraph{Example.}
\begin{verbatim}
make MODE=working MEDIA=screen ENGINE=xe VERBOSE=true
\end{verbatim}

\section{Extending the Makefile}

Advanced users may extend the Makefile with new automation routines.  
Custom targets can be added without interfering with the core compilation logic.

\paragraph{Example: Spell Check.}
\begin{verbatim}
spell:
    aspell --mode=tex check 2-MainMatter/chapter1.tex
\end{verbatim}

\paragraph{Example: PDF/A Conversion.}
\begin{verbatim}
pdfa:
    gs -dPDFA=2 -dBATCH -dNOPAUSE \
       -sDEVICE=pdfwrite \
       -sOutputFile=template_PDFA.pdf template.pdf
\end{verbatim}

\paragraph{Example: Increment Version Tag.}
\begin{verbatim}
version:
    @echo "Updating version header..."
    @date "+%Y/%m/%d" | xargs -I{} sed -i \
        's/\(ProvidesClass{novathesis}\[.*v\)[0-9.]*\( NOVA Thesis Template\]/\1$(VERSION)\2/' \
        NOVAthesisFiles/novathesis.cls
\end{verbatim}

\section{Cross-Platform Compatibility}

The Makefile has been tested on:
\begin{itemize}
  \item macOS (MacTeX, GNU~Make 3.81 or later);
  \item Linux (TeX~Live, GNU~Make 4.3 or later);
  \item Windows (MiKTeX or TeX~Live via WSL/Git Bash).
\end{itemize}

If the system lacks GNU~Make, compilation may still be performed manually:
\begin{verbatim}
latexmk -xelatex template
biber template
makeglossaries template
latexmk -xelatex template
\end{verbatim}

\section{Typical Build Scenarios}

\paragraph{Quick Draft.}
\begin{verbatim}
make MODE=working preview
\end{verbatim}

\paragraph{Final Submission (paper version).}
\begin{verbatim}
make MODE=final MEDIA=paper xe
make package
\end{verbatim}

\paragraph{Digital Archive (PDF/A).}
\begin{verbatim}
make MODE=final MEDIA=screen xe
make pdfa
\end{verbatim}

\paragraph{Continuous Integration Test.}
\begin{verbatim}
make ci
\end{verbatim}

\section{Summary}

The NOVAthesis Makefile provides a standardized, reproducible build environment across platforms and institutional contexts.  
By using the provided targets, authors can automate all compilation, packaging, and validation steps without manual intervention.  
For maintainers, the Makefile forms the foundation of continuous integration and long-term reproducibility within the NOVAthesis development workflow.

