%!TEX root = ../template.tex

\chapter{Build Environment Reference}
\label{chap:build-environment}

\section{Overview}

This appendix provides a reference for system administrators and advanced users who deploy NOVAthesis across multiple environments.  
It lists the essential executables, package dependencies, environment variables, and diagnostic tools required for a reproducible build.  
It is intended as a concise technical guide complementing the installation and troubleshooting chapters.

\section{Core Requirements}

A complete NOVAthesis build environment consists of three layers:

\begin{enumerate}
  \item A \LaTeX{} distribution (TeX~Live, MacTeX, or MiKTeX) including \texttt{latexmk} and \texttt{biber};
  \item A system shell environment capable of executing \texttt{make};
  \item A minimal set of additional utilities (e.g., Ghostscript, Poppler) for PDF validation and post-processing.
\end{enumerate}

Table~\ref{tab:required-components} summarizes the standard dependencies.

\begin{table}[h]
\centering
\caption{Required Components for NOVAthesis Compilation}
\label{tab:required-components}
\begin{tabular}{llp{6cm}}
\toprule
\textbf{Component} & \textbf{Example Command} & \textbf{Purpose} \\
\midrule
\LaTeX{} distribution & \texttt{texlive} / \texttt{mactex} / \texttt{miktex} & Core typesetting engine and package repository \\
\texttt{latexmk} & \texttt{latexmk --version} & Automates multi-pass compilation \\
\texttt{biber} & \texttt{biber --version} & Bibliography processor (preferred over \texttt{bibtex}) \\
\texttt{make} & \texttt{make -v} & Executes the Makefile targets \\
\texttt{ghostscript} & \texttt{gs --version} & PDF post-processing (optional for validation) \\
\texttt{bash} & \texttt{bash --version} & Required by the Makefile on Windows systems \\
\bottomrule
\end{tabular}
\end{table}

\paragraph{Recommended versions (as of 2025).}
\begin{itemize}
  \item TeX~Live 2024 or newer (full installation);
  \item MacTeX 2024 or newer;
  \item MiKTeX 24.3 or newer (with automatic package installation enabled);
  \item Biber 2.20 or later.
\end{itemize}

\section{File Structure Dependencies}

The NOVAthesis build system assumes the following directory layout (also summarized in Chapter~\ref{chap:document-structure}):

\begin{verbatim}
.
├── 0-Config/         % Configuration files and metadata
├── 1-FrontMatter/    % Abstracts, acknowledgments, etc.
├── 2-MainMatter/     % Chapters
├── 3-BackMatter/     % Appendices and annexes
├── 4-Bibliography/   % .bib files
├── 5-Figures/        % Figures and images
├── NOVAthesisFiles/  % Internal assets (do not modify)
├── Makefile          % Build automation
└── template.tex      % Main entry point
\end{verbatim}

The build scripts depend on these relative paths.  
Renaming directories or moving configuration files outside this structure will cause build failures unless the Makefile rules are updated accordingly.

\section{External Executables and Environment Variables}

\subsection{Executables}

The Makefile and \texttt{latexmk} use several auxiliary executables internally:

\begin{itemize}
  \item \texttt{pdflatex}, \texttt{xelatex}, or \texttt{lualatex};
  \item \texttt{biber} or \texttt{bibtex};
  \item \texttt{makeindex} or \texttt{xindy} (for glossaries);
  \item \texttt{pdfcrop}, \texttt{pdfinfo}, \texttt{pdfjam} (optional, for post-processing);
  \item \texttt{kpsewhich} (used to locate style files).
\end{itemize}

\subsection{Environment Variables}

The following environment variables influence the build process:

\begin{description}
  \item[\texttt{PATH}] — Must include directories containing \texttt{latexmk}, \texttt{biber}, and the \LaTeX{} executables.
  \item[\texttt{ENGINE}] — Overrides the default compilation engine.  
  Example: \texttt{ENGINE=xelatex}.
  \item[\texttt{MODE}] — Sets the build mode (\texttt{working}, \texttt{provisional}, \texttt{final}).
  \item[\texttt{SCHOOL}] — Overrides the institutional preset.
  \item[\texttt{LATEXMKOPTS}] — Appends custom options to the \texttt{latexmk} command line.
  \item[\texttt{BIBER}] — Explicit path to the \texttt{biber} executable, if not in PATH.
\end{description}

\paragraph{Example.}
\begin{verbatim}
export ENGINE=xelatex
export MODE=final
make
\end{verbatim}

\section{Package Dependencies}

NOVAthesis relies on a standard set of \LaTeX{} packages.  
Most are included in full TeX~Live or MacTeX distributions.  
Administrators installing a minimal system must ensure the following collections are present:

\begin{itemize}
  \item \texttt{collection-latexrecommended}
  \item \texttt{collection-fontsrecommended}
  \item \texttt{collection-latexextra}
  \item \texttt{collection-bibtexextra}
  \item \texttt{collection-langportuguese} (for bilingual output)
  \item \texttt{collection-luatex} and \texttt{collection-xetex}
\end{itemize}

Additionally, the following individual packages are directly used by the class:
\texttt{memoir}, \texttt{fontspec}, \texttt{babel}, \texttt{csquotes}, \texttt{biblatex}, \texttt{graphicx}, \texttt{xcolor}, \texttt{booktabs}, \texttt{hyperref}, \texttt{cleveref}, \texttt{fancyvrb}, \texttt{etoolbox}, and \texttt{titlesec}.

\section{Filesystem and Encoding Settings}

\begin{itemize}
  \item All text files should be encoded in UTF-8 without BOM.  
  Legacy encodings (Latin-1 or Windows-1252) can cause errors under XeLaTeX or LuaLaTeX.
  \item File and folder names should contain only lowercase alphanumeric characters and underscores.
  \item Avoid spaces or diacritics in filenames, especially on Overleaf or network drives.
\end{itemize}

\section{Compiler-Specific Notes}

\subsection{pdf\LaTeX}

Advantages:
\begin{itemize}
  \item Fast and stable;
  \item Fully portable;
  \item Compatible with most institutional fonts if packaged as Type~1.
\end{itemize}

Limitations:
\begin{itemize}
  \item No direct OpenType or system font support;
  \item Limited multilingual typesetting;
  \item No native transparency in vector graphics.
\end{itemize}

\subsection{XeLaTeX}

Advantages:
\begin{itemize}
  \item Native Unicode support;
  \item Direct access to system fonts;
  \item Simple multilingual configuration.
\end{itemize}

Limitations:
\begin{itemize}
  \item Slightly slower compilation time;
  \item Requires all fonts to be installed locally;
  \item Some packages (e.g., \texttt{microtype}) offer reduced functionality.
\end{itemize}

\subsection{LuaLaTeX}

Advantages:
\begin{itemize}
  \item Modern engine with advanced microtypography;
  \item Fully compatible with \texttt{fontspec};
  \item Extensible through Lua scripting.
\end{itemize}

Limitations:
\begin{itemize}
  \item Requires more memory;
  \item Slightly different font metrics compared to XeLaTeX;
  \item Marginally slower on small projects.
\end{itemize}

\section{Build Modes}

NOVAthesis distinguishes three build modes, controlled by the \texttt{mode} key or the environment variable \texttt{MODE}.

\begin{description}
  \item[\texttt{working}] — Draft mode; includes todos, colored hyperlinks, and visible overfull boxes.
  \item[\texttt{provisional}] — Near-final mode; hyperlinks are colored, but minor draft indicators are suppressed.
  \item[\texttt{final}] — Production mode; black hyperlinks, all debugging features disabled, and layout frozen.
\end{description}

Each mode affects internal variables that govern the behavior of \texttt{hyperref}, watermark visibility, and header/footer annotations.

\section{Testing and Validation}

\subsection{Verifying the Toolchain}

Run the following commands to confirm availability of required tools:

\begin{verbatim}
latexmk -version
biber --version
kpsewhich novathesis.cls
\end{verbatim}

\subsection{Test Compilation}

Perform a full rebuild:

\begin{verbatim}
make clean-all
make xe
\end{verbatim}

Verify that:
\begin{itemize}
  \item The PDF is generated without errors;
  \item The log file contains no “undefined reference” or “missing citation” warnings;
  \item Fonts are embedded (check with \texttt{pdfinfo -f template.pdf}).
\end{itemize}

\section{Administrator Checklist}

Before deploying NOVAthesis on institutional or shared systems, confirm:

\begin{enumerate}
  \item Full \LaTeX{} distribution installed and updated;
  \item \texttt{latexmk}, \texttt{biber}, and \texttt{make} available in all user PATHs;
  \item UTF-8 locale configured by default;
  \item Write permissions enabled in project directories;
  \item Sufficient disk space (minimum 2~GB per user);
  \item The \texttt{NOVAthesisFiles/} folder is intact and unmodified.
\end{enumerate}

\section{Diagnostics for Technical Reviewers}

When diagnosing a user’s build failure, request the following artifacts:
\begin{itemize}
  \item \texttt{template.log} — compilation log;
  \item \texttt{template.blg} — bibliography log;
  \item \texttt{template.bcf} — bibliography control file (for Biber);
  \item \texttt{0-Config/1\_novathesis.tex} — core configuration;
  \item Output of \texttt{latexmk -version} and \texttt{biber --version}.
\end{itemize}

Compare with the reference build in Table~\ref{tab:required-components} to identify version discrepancies.

\section{Minimal Diagnostic Script}

Administrators may use the following shell script to validate an installation:

\begin{verbatim}
#!/usr/bin/env bash
echo "=== NOVAthesis Build Environment Diagnostic ==="
for cmd in latexmk biber xelatex lualatex pdflatex make kpsewhich; do
    if ! command -v $cmd &>/dev/null; then
        echo "Missing executable: $cmd"
    else
        echo "$cmd OK ($(command -v $cmd))"
    fi
done
echo "TeX distribution version:"
tex --version | head -n 1
\end{verbatim}

\section{Version Identification}

Every NOVAthesis release includes a version header within \texttt{novathesis.cls}:
\begin{verbatim}
\ProvidesClass{novathesis}[2025/04/07 v7.6.0 NOVA Thesis Template]
\end{verbatim}

Administrators maintaining multiple installations should reference this value when reporting bugs or submitting compatibility updates.

\section{Archival and Long-Term Reproducibility}

For archival or institutional repository submissions:
\begin{itemize}
  \item Include the compiled PDF, the \texttt{0-Config/} directory, and the bibliography files;
  \item Document the TeX distribution version and NOVAthesis release number;
  \item Optionally include a checksum (e.g., SHA-256) of the final PDF;
  \item Avoid distributing auxiliary files or temporary artifacts.
\end{itemize}

\section{Summary}

This reference consolidates all technical information necessary for deploying and maintaining NOVAthesis in academic and research environments.  
It ensures that administrators can provide consistent toolchains, detect configuration problems quickly, and reproduce published theses exactly as originally compiled.  
A properly configured environment with verified dependencies guarantees long-term stability, compatibility with future \LaTeX{} releases, and compliance with institutional archiving requirements.
